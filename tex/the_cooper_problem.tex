\section{The Cooper problem}

We consider a non-interacting Fermi-sea $\ket{\Phi_0}$, and two additional electrons. These two electrons do not interact with the Fermi-sea. They do however interact with each other. Let us now specify the way in which they interact. The initial states of the two electrons are taken to be $\ket{k,\uparrow}$ and $\ket{-k, \downarrow}$, such that a non-interacting two-particle state can be denoted by $\ket{k, \uparrow; -k, \downarrow}$. Short hand notation for this will be $\ket{k, -k}$, with the understanding that the electrons have opposite spins. While this choice of initial states may look a bit wierd, this is what we will work with, its justification will be made clear later. (Hint: go back and study the physical explanation for attractive phonon-mediated electron-electron interaction). The situation may be illustrated as in \cref{fig:cooper_situation}.

\begin{figure}
	\centering
	\begin{tikzpicture}[scale = 1]
	
	\coordinate (a) at (2, 2);
	\coordinate (k) at (1.7, 1.4);
	\coordinate (q) at (-2.4, 1);
	\coordinate (kprime) at (1.8, -1.1);
	
	\draw[thick,fill, blue!50!gray,  opacity =0.5] (a) circle (1.9);
	\draw[fill] (a) circle (0.05);
	\draw[fill] ($(a) + (k)$) circle (0.05);
	\draw[fill] ($(a) - (k)$) circle (0.05);
	\draw[fill] ($(a) + (k) + (q)$) circle (0.05);
	\draw[fill] ($(a) - (k) - (q)$) circle (0.05);
	
	
	\draw[thick,dashed] (a) --  ++(k)node[above right]{\large $\ket{\vb k, \uparrow}$};
	\draw[thick,  -] (a)++(k) --  ++(q) node[above]{\large $\ket{\vb k', \uparrow}$};
	
	
	\draw[thick,dashed] (a) --  ($(a) - (k)$) node[below left]{\large $\ket{-\vb k, \downarrow}$};
	\draw[thick] ($(a) - (k)$)  ++($(0,0) -(q)$) node[below]{\large $\ket{-\vb k', \downarrow}$};
	\draw[thick, dashed] ($(a) - (k)-(q)$) -- ($(a) + (k) + (q)$);
	
	\draw[thick] (-1,2) -- (5,2);
	\draw[thick] (2,-1) -- (2,5);
	
	\node at (4.2, 1) {\Large $\sim \ket{\Phi_0}$};
\end{tikzpicture}
	\caption{The scattering situation. }
	\label{fig:cooper_situation}
\end{figure}

The interaction among the electrons is now such that it scatters them into a new two-particle state $\ket{k', -k'}$, see illustration above. The interaction that causes such a scattering is denoted $V$. This interaction is assumed to be operative within a thin shell of width $\omega_0$ around the Fermi-surface. 

\begin{equation}
\Ha_0 \ket{k,-k} = \ep_k \ket{k,-k}
\end{equation}

$\ep_k$: Kinetic energy of the two added electrons. $\Ha_0$: Hamiltonian with no $V_{kk'}$. \\

Exact two-particle state with interactions: $\ket{1,2}$. 

\begin{equation}
\ket{1,2} = \sum_{k'} a_{k'}\ket{k', -k'},
\end{equation}

where $a_{k'}$ must be determined. 

\begin{equation}
(\Ha_0 + V_{eff})\ket{1,2} = E\ket{1,2}
\end{equation}

$E$: Exact two-particle energy for this problem. 

\begin{align*}
(\Ha_0 + V_{eff})\sum_{k'} a_{k'}\ket{k', -k'} = \sum_{k'} a_{k'}(\ep_{k'} + V_{eff})\ket{k', -k'} = \sum_{k'} a_{k'}E\ket{k', -k'}.
\end{align*}

Multiplying with $\bra{k,-k}$ and use orthonormality $\delta_{k,k'} = \braket{k,-k}{k',-k'}$. 

\begin{align*}
a_k \ep_k + \sum_{k'} \bra{k, -k} V_{eff} \ket{k', -k'} = a_k \ep_k + \sum_{k'} V_{k,k'} = a_k E
\end{align*}

\begin{align*}
V_{k,k'} = \begin{cases} -V & k,k' \in \Omega \\
0 & k,k' \notin \Omega \end{cases} 
\end{align*}

where $\Omega$ denotes a region in k-space in the close vicinity of the Fermi-surface ("thin shell"). The Schroedinger equation now reads 

\begin{align*}
a_k(\ep_k - E) = -\sum_{k' \in \Omega}a_{k'}V_{k,k'} = \sum_{k' \in \Omega}V a_{k'}
\end{align*}

where we have inserted the expression for $V_{k,k'}$ in the $\Omega$-region in k-space. We make this summation explicit by including Heaviside step-functions. 

\begin{align*}
a_k (\ep_k - E) \Theta(2\omega_0 - \abs{\ep_k - 2\ep_F}) = V \sum_{k'} a_{k'} \Theta(2\omega_0 - \abs{\ep_k' - 2\ep_F}) \Theta(2\omega_0 - \abs{\ep_k - 2\ep_F}).
\end{align*}

The two step-functions on the r.h.s originate with the fact that both $k$ and $k'$ in $V_{k,k'}$ must be within this energy shell around the Fermi-surface. The step-functions that depend on $k$ cancel on both sides of the equation. Thus, we have 

\begin{align*}
\sum_{k'} a_{k'} \Theta(2\omega_0 - \abs{\ep_k' - 2\ep_F}) = K_1, 
\end{align*}

constant and independent of $k$. 

\begin{align*}
a_k = \frac{K_1}{\ep_k - E}
\end{align*}

$a_k$ depends on $k$ only via $\ep_k$, so we can convert the k-sum into an energy sum: 

\begin{align*}
\sum_{k'}f(\ep_{k'}) = \sum_{k'} \int_{-\infty}^{\infty} \dd \ep  \delta(\ep - \ep_{k'})f(\ep) = \int \dd \ep N(\ep) f(\ep)
\end{align*}

NB: recall that $\ep$ is a two-particle kinetic energy. 

\begin{align*}
a(\ep)(\ep - E) = V\int_{\abs{\ep' - 2\ep_F} < 2\omega_0} \dd \ep' a(\ep') N(\ep') 
\end{align*}
\begin{align*}
a(\ep) = \frac{K_1}{\ep - E} \implies K_1 = V\int_{\abs{\ep' - 2\ep_F} < 2\omega_0} \dd \ep' \frac{K_1}{\ep' - E} N(\ep') 
\end{align*}

Thus, the unknown factor $K_1$ drops out. Furthermore, the energy shell is thin, and we assume $N(\ep)$ is a slowly varying function of $\ep$ around the , $\ep \approx 2\ep_F$. 

\begin{align*}
1 \approx \overbrace{VN(\ep_F)}^{\equiv \lambda} \int_{2\ep_F}^{2\ep_F + 2\omega_0} \dd \ep' \frac{1}{\ep' - E}
\end{align*}

$\lambda > 0$, by definition. \\
\begin{align*}
\frac{1}{\lambda} = \ln \abs{\frac{2\ep_F + 2\omega_0 -E}{2\ep_F - E}}
\end{align*}

Now define the two particle binding-energy $\Delta \equiv 2\ep_F - E$

Then, 

\begin{align*}
\frac{1}{\lambda} = \ln \left(1 + \frac{2\omega_0}{\Delta}\right)
\end{align*}

$\lambda > 0$: Solution requires that $\Delta > 0 \implies E < 2\ep_F$!

Solve for $\Delta$: 

\begin{align*}
\frac{2\omega_0}{\Delta} = \e^{\frac{1}{\lambda}} - 1 \\
\Delta = 2\omega_0 \frac{1}{\e^{\frac{1}{\lambda}} - 1 } \\
\lambda \ll 1: \Delta \approx 2\omega_0 \e^{-\frac{1}{\lambda}}.
\end{align*}

Had we cept $\hbar$ explicitly in the calculations, we would have found 

\begin{equation}
\label{eq:cooper_delta}
\Delta = 2\hbar \omega_0 \e^{-\frac{1}{\lambda}}. 
\end{equation}

We next comment on $E < 2\ep_F$. At first glance, this would seem to violate the Pauli-principle, since it looks like we have the two added electrons now residing inside the Fermi-sea. In fact, there is no violation of the Pauli-principle, for the following reason. The bound state must be viewed as an entity, not as two individual electrons. We may think about this as a state created by a creation operator 

\begin{equation}
\bd_k = \cd_{k \uparrow} c_{-k \downarrow}
\end{equation}

and corresponding destruction operator 

\begin{equation}
b_k = \cd_{-k \downarrow} c_{k \uparrow}.
\end{equation}

These operators do not obey fermionic anti-commutation relations. Therefore, this composite particle is not a fermion. Note that by forming this pair, called a Cooper-pair, the electrons have gotten rid of the severe limitations posed by the Pauli-principle, and this composite state is allowed to reside inside the Fermi-sea. A few remarks are in order: 

\begin{itemize}
	\item The bound state energy \\ $\Delta = 2\hbar \omega_0 \e^{-\frac{1}{\lambda}}$ \\ This means that $\Delta > 0$ is a quantum effect, since it requires $\hbar \neq 0$. 
	\item $\Delta$ is exponentially sensitive to $\lambda = VN(\ep_F)$. Note that, in addition to V, we must have a non-zero density of states on the Fermi-surface. Thus, a Fermi-surface is requried to get Cooper-pairs. 
	\item These electron-pairs are bound states in momentum-space, not in real space! 
	\item Note how $\Delta$ depends on $\lambda$ . As $\lambda \to 0$, $\Delta$ features an essential singularity. Such a result could not have been obtained in perturbation theory to any finite order. An attractive electron-electron interaction is a singular perturbation. 
	\item Thermal effects: It stands to reason that the bound state we have found will be dissocated at some temperature. This temperature will be $k_B T_0 \sim \Delta$. Above this temperature, no Cooper-pairs will exist. 
	\item Note also the important fact that the binding of two electrons into a Cooper-pair (a composite non-fermionic entity) does not rely on a spesific mechanism for producing an attractive electron-electron interaction. Thus, the consept is quite general in any fermionic system with a non-zero density of states on the Fermi-surface. 
	\item Note also that we are careful in not saying that Cooper-pairs are bosons, they are in fact not! Elementary particles are either bosons or fermions, but Cooper-pairs is not an elementary particle, obviously. 
	\item The problem just considered is admittedly quite artificially, only two electrons in a thin shell around the Fermi-surface interact. What if we include interactions also among all electrons within a thin shell around the Fermi-surface? 
\end{itemize}

In that case, we would have a real many-body problem to solve, with an interaction that cannot be treated in perturbation theory (lesson from Cooper-problem). This is the problem we will next, and it leads to a microscopic theory of superconductivity, the phenomena that a metal looses all electrical resistance below a certain temperature. The binding energy $\Delta$ of a Cooper-pair turns into a gap in the excitation spectrum of electrons close to the Fermi-surface. This protects electrons from scattering, leading to zero resistivity. This happens at a temperature $\sim$ gap at zero temperature. 