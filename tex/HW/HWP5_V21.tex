\documentclass{article}
%  NB: x er en stylefil!
\setlength{\textwidth}{16.0cm}
\setlength{\textheight}{22.0cm}
\setlength{\hoffset}{-1.5cm}
\setlength{\voffset}{-1.0cm}
\setlength{\topskip}{0cm}
\setlength{\parskip}{0cm}
\usepackage{amsmath}
%\usepackage{mathtools}
\begin{document}
\noindent{\Large\bf Department of Physics, NTNU}\\

\centerline{\Large Homework 5}
\centerline{\Large  TFY4210/FY8302 Quantum  theory of solids}
\centerline{\Large Spring 2021.}
\normalsize
\ \\
\ \\
\underline{\large\bf Problem 1 }\\
\ \\
\ \\
Use the results we derived in class to compute the low-temperature thermal corrections to the sublattice magnetization in the isotropic Heisenberg quantum antiferromagnet, in arbitrary dimensions $d$. Low temperature here means that $T \ll J$.   
\ \\
\ \\
\underline{\large\bf Problem 2 }\\
\ \\
\ \\
In this problem, we will study the coupling between a system of itinerant electrons (i.e. electrons that can move around), modelled as a tight-binding model with hopping on a lattice and nearest neighbor hopping, and a system of localized spins (spin $1/2$) ${\bf S}_i$ which are located on the same lattice as the electrons can hop. We denote the electron spins as 
\begin{eqnarray}
{\bf s}_i = \frac{1}{2} c^{\dagger}_{i \alpha} {\vec  \sigma}_{\alpha \beta} c_{i \beta} \nonumber
\end{eqnarray}
where $c^{\dagger}_{i \sigma}, c_{i \sigma}$ are creation and destruction operators for the electrons, a summation convention is implicit on $\alpha$ and $\beta$, and $\vec \sigma = (\sigma_x,\sigma_y,\sigma_z)$ are the Pauli-matrices. 
\ \\
\ \\
It is assumed that  the electron-spins interact with the localized spins through an exchange coupling of the type 
$-J_{sd}  \sum_i {\bf S}_i \cdot {\bf s}_i$. The subscript $sd$ refers to the fact that this type of model is often used to describe itinerant electrons interacting with localized spins (a metallic magnet), where the dominant orbital content of the electron band is a Wannier-orbital with $s$-wave symmetry, while the dominant orbital content of the localized spins is a Wannier-orbital with  $d$-wave symmetry.  
\ \\
\ \\
The Hamiltonian of the system is given by 
\begin{eqnarray}
H & = & H_{\rm{el}} + H_{\rm{spin}} + H_{\rm{el-spin}} \nonumber \\ 
H_{\rm{el}} & = & - t \sum_{\langle i,j \rangle, \sigma } c^{\dagger}_{i \sigma}c_{j \sigma} - \mu \sum_{i,\sigma} c^{\dagger}_{i \sigma}c_{i \sigma} \nonumber \\
H_{\rm{spin}} & = & -J \sum_{\langle i,j \rangle } {\bf S}_i \cdot {\bf S}_j \nonumber \\ 
H_{\rm{el-spin}} & = & - J_{sd} \sum_i {\bf S}_i \cdot {\bf s}_i \nonumber 
\end{eqnarray}
You may assume that $(J,J_{sd}) > 0$, and that the localized spins are almost completely ferromagnetically magnetically ordered, such that a truncation of the Holstein-Primakoff transformation for ${\bf S}_i$ to lowest order in magnon-operators is justified. A spin-flip ``up'' operator for the electrons is given by 
${s}_{i+} = c^{\dagger}_{i \uparrow} c_{i \downarrow}$, while a spin-flip ``down'' operator for the electrons is given by 
${s}_{i-} = c^{\dagger}_{i \downarrow} c_{i \uparrow}$ 
(for details, see the lectures notes on the derivation of the Heisenberg model from the Hubbard-model).   
\ \\
\ \\
{\bf a)} Show that, to lowest order in the magnon-operators introduced in the Holstein-Primakoff-transformation,  the Hamiltonian may be written on the form
\begin{eqnarray}
H & = & E_0 + 2JS \sum_{\langle i,j, \rangle} \left[ a^{\dagger}_i a_i -  a^{\dagger}_i a_j  \right]
- t \sum_{\langle i,j \rangle, \sigma } c^{\dagger}_{i \sigma}c_{j \sigma} - \mu \sum_{i,\sigma} c^{\dagger}_{i \sigma}c_{i \sigma}
- J_{sd}  S \sum_{i,\sigma} \sigma ~c^{\dagger}_{i \sigma}c_{i \sigma} \nonumber \\
& + & J_{sd} \sum_{i,\sigma} \sigma ~
a^{\dagger}_{i} a_{i} c^{\dagger}_{i, \sigma} c_{i,\sigma} - \frac{J_{sd}}{2} \sum_i \sqrt{2S}\left[ a_i  ~c^{\dagger}_{i \downarrow} c_{i \uparrow}+ a^{\dagger}_{i} ~ c^{\dagger}_{i \uparrow} c_{i \downarrow}  \right] \nonumber
\end{eqnarray}
\ \\
\ \\
{\bf b)} Give a physical interpretation of the last term in the first line, and explain its presence. 
\ \\
\ \\
{\bf c)} Give a physical interpretation of the difference in spin-structure of the two terms in the second line.
\ \\
\ \\
{\bf c)} Express the Hamiltonian in terms of Fourier-transformed magnon- and electron-operators. 
\ \\
\ \\
{\bf d)} Explain how the above Hamiltonian effectively may contain interactions between electrons. (Hint: It may be helpful to draw on an analogy with electron-phonon coupling, and consider diagrammatic representations of the electron-magnon coupling like we did in class for electron-phonon coupling). 
\ \\
\ \\
{\bf e)} In the above Hamiltonian, there is the nearest-neighbor spin-interaction term $-J \sum_{\langle i,j \rangle} {\bf S}_i \cdot {\bf S}_j$. Explain how the above Hamiltonian generates additional, longer-ranged, interactions between spins ${\bf S}_i$ on different lattice sites. (Hint: Try to give a pictorial representation of these interactions in the same way that we used pictures to represent interactions among electrons via phonons in class.)


\end{document}

