\documentclass{article}
%  NB: x er en stylefil!
\setlength{\textwidth}{16.0cm}
\setlength{\textheight}{22.0cm}
\setlength{\hoffset}{-1.5cm}
\setlength{\voffset}{-1.0cm}
\setlength{\topskip}{0cm}
\setlength{\parskip}{0cm}
%\usepackage{axodraw}
\usepackage[utf8]{inputenc}
\usepackage{amsmath}
\usepackage{feynmf}
%\usepackage{tikz-feynman}
\usepackage{tikz}
\usepackage[compat=1.1.0]{tikz-feynman}
\newcommand{\comma}{\mathbin{{,}}}
\usepackage{mathtools}
\usetikzlibrary{snakes}
\begin{document}
\noindent{\Large\bf Department of Physics, NTNU}\\

\centerline{\Large Homework 8}
\centerline{\Large  TFY4210/FY8916 Quantum theory of solids}
\centerline{\Large Spring 2021.}
\normalsize
\ \\
\ \\
\underline{\large\bf Problem 1 }\\
\ \\
\ \\
The creation and destruction operator of a Cooper-pair is given by 
\begin{eqnarray}
b_{\bf k} & =  & c_{-{\bf k}, \downarrow} c_{{\bf k}, \uparrow} \nonumber \\  
b^{\dagger}_{\bf k} & =  & c^{\dagger}_{{\bf k}, \uparrow} c^{\dagger}_{-{\bf k}, \downarrow} \nonumber  
\end{eqnarray}
\ \\
\ \\
{\bf a) } Compute the commutators $\left[ b_{\bf k}, b^{\dagger}_{{\bf k}^{\prime}} \right]$  and 
$\left[ b_{\bf k}, b_{{\bf k}^{\prime}} \right]$, and the anti-commutator  $\left\{ b_{\bf k}, b_{{\bf k}^{\prime}} \right\}$. 
Compare your results with what you find if the operators had been boson-operators. 
\ \\
\ \\
{\bf b)} Imagine that you had two different condensed matter systems, one with a relatively strong attraction between electrons, and one with a relatively weak attraction between electrons.  Which one of these two system would you think a description of Cooper-pairs as bosons would be the best approximation?  
\ \\
\ \\
\ \\
\underline{\large\bf Problem 2 }\\
\ \\
\ \\
In Problem 1 we considered creation an destruction operators for a Cooper-pair. Consider now a bosonic {\it pair-fluctuation field} $\phi({\bf q}, \omega)$ which can split into two electrons, and two electrons can recombine into the field $\phi({\bf q}, \omega)$. 
\ \\
\ \\
A Feynman-diagram for such a process is given in the Figure below. 
\ \\
\ \\
\begin{center}
\begin{tikzpicture}
\node at (11, 0) {$= \chi_0({\bf q}, \omega)  ~  K({\bf q},\omega)  ~ \chi_0({\bf q}, \omega) $};
    \begin{feynman}
        \vertex (a);
        \vertex[right=2cm of a] (b);
        \vertex[right=4cm of b] (c);
        \vertex[right=2cm of c] (d);
        \diagram*{(a)--[charged boson, edge label = \(q \comma \omega\)] (b),
        (b) -- [fermion, quarter left, edge label= \(q - k \comma \omega - \omega'\)] (c), 
        (b) -- [fermion, quarter right, edge label' = \(k \comma \omega'\)] (c),
        (c) -- [charged boson, edge label = \( q \comma \omega \)] (d)  
        };
    \end{feynman} 
\end{tikzpicture}
\end{center}
\ \\
\ \\
The wavy line is the Green's function for the free field $\phi({\bf q},\omega)$, denoted $\chi_0({\bf q}, \omega)$.
Denote the bubble-diagram by $K({\bf q},\omega)$. Consider now a Dyson-equation for the Green's function 
$\chi({\bf q}, \omega)$ for the pairing field
\ \\
\ \\
\begin{eqnarray}
\chi({\bf q}, \omega)^{-1} & = & \chi_0({\bf q}, \omega)^{-1} - K({\bf q},\omega) \nonumber \\
\chi_0({\bf q}, \omega)^{-1} & = & V \nonumber
\end{eqnarray} 
Here, $V=$ is a constant attractive interaction that works in a thin shell around the Fermi-surface.  
\ \\
\ \\
{\bf a)} Use the Feynman-rules to compute the integral over $\omega^{\prime}$ in the bubble-diagram, taking care to express your answer in terms of $T=0$ Fermi-distribution $\theta(\varepsilon_F-\varepsilon_{\bf k})$ and its complementary  distribution $\theta(\varepsilon_{\bf k} - \varepsilon_F )$, and corresponding quantities with 
${\bf k} \to {\bf k} +{\bf q}$.
\ \\
\ \\
{\bf b)} Generalize this to finite temperatures $T>0$ by replacing the $\theta$-functions by finite-temperature Fermi-distributions and simplify the expression by letting ${\bf q} \to 0$.   
\ \\
\ \\
{\bf c)}$\chi({\bf q}, \omega)$ may be given a physical interpretation as a {\it pair-susceptibility}, i.e. the ability an electron system has to form pairs of electrons, Cooper-pairs. The divergence of this susceptibility indicates an instability of the Fermi-sea of electrons. Find the temperature at which this instability takes place by explicitly computing the remaining ${\bf k}$-sum. (Hint: Consider the case ${\bf q}=0$ and convert the sum over ${\bf k}$ to an energy integral and approximate the single-particle density of states by its value on the Fermi-surface.)
\ \\
\ \\
Note how very differently the particle-particle bubble $K({\bf q},\omega)$ behaves from the particle-hole bubble
$\Pi({\bf q},\omega) $ we considered in class.
\begin{center}
\begin{tikzpicture}
\node at (11, 0) {$= D_0({\bf q}, \omega)  ~  \Pi({\bf q},\omega)  ~ D_0({\bf q}, \omega) $};
    \begin{feynman}
        \vertex (a);
        \vertex[right=2cm of a] (b);
        \vertex[right=4cm of b] (c);
        \vertex[right=2cm of c] (d);
        \diagram*{(a)--[charged boson, edge label = \(q \comma \omega\)] (b),
        (b) -- [fermion, quarter left, edge label= \(q + k \comma \omega + \omega'\)] (c), 
        (c) -- [fermion, quarter left, edge label' = \(k \comma \omega'\)] (b),
        (c) -- [charged boson, edge label = \( q \comma \omega \)] (d)  
        };
    \end{feynman} 
\end{tikzpicture}
\end{center}
There, we found that $\lim_{{\bf q} \to 0} \Pi({\bf q},\omega=0) = - 2 N(\varepsilon_F)$. This is essentially $T$-independent. 



\end{document}

