\documentclass{article}
%  NB: x er en stylefil!
\setlength{\textwidth}{16.0cm}
\setlength{\textheight}{22.0cm}
\setlength{\hoffset}{-1.5cm}
\setlength{\voffset}{-1.0cm}
\setlength{\topskip}{0cm}
\setlength{\parskip}{0cm}
\usepackage{amsmath}
%\usepackage{mathtools}
\begin{document}
\noindent{\Large\bf Department of Physics, NTNU}\\

\centerline{\Large Homework 4}
\centerline{\Large  TFY4210/FY8302 Quantum theory of solids}
\centerline{\Large Spring 2021.}
\normalsize
\ \\
\ \\
\underline{\large\bf Problem 1 }\\
\ \\
\ \\
In class, we have seen how to compute the magnon spectrum of the isotropic ferromagnetic Heisenberg quantum spin model
\begin{eqnarray}
H & = & - J \sum_{\langle i,j\rangle} {\bf S} _i \cdot {\bf S}_j \nonumber \\
  & = & -    J \sum_{\langle i,j\rangle} \left[S_{iz}  S_{jz} + \frac{1}{2}   \left(S_{i+} S_{j-} + S_{j+} S_{i-} \right)  \right] \nonumber
\end{eqnarray}  
by using the Holstein-Primakoff transformation and calculating to quadratic order in magnon-operators. 
\ \\
\ \\
{\bf a)} Generalize the results of {\bf a)} to the case with arbitrary-ranged {\it ferromagnetic}, but isotropic in spin-space, spin-interactions $J_{ij}>0$, i.e. the Hamiltonian is given by 
\begin{eqnarray}
H  & = & -  \sum_{i,j} J_{ij} ~ \left[S_{iz}  S_{jz} + \frac{1}{2}   \left(S_{i+} S_{j-} + S_{j+} S_{i-} \right)  \right] \nonumber
\end{eqnarray}  
You may assume that $J_{ij}= J({\bf r}_i-{\bf r}_j)$, where ${\bf r}_i$ is the position of lattice site $i$. Find the expression for the magnon-spectrum in this case. 
\ \\
\ \\
{\bf b)} In class, we have seen that there are no quantum fluctuations in the isotropic ferromagnetic Heisenberg-model, while we do get quantum-fluctuations through squeezing of magnons in the ground state of isotropic antiferromagnetic case. 
\ \\
\ \\
Consider now a generalization of the ferromagnetic Heisenberg-model with nearest-neighbor interactions to the case with spin-space anisotropy
\begin{eqnarray}
H = - \sum_{\langle i,j \rangle} \left[ J S_{iz} S_{jz} + J_x S_{ix} S_{jx} + J_y S_{iy} S_{jy} \right]. \nonumber
\end{eqnarray} 
(Such spin-space anisotropy could, for instance, originate with spin-orbit coupling in the system.) 
In this problem, we will assume that $(J,J_x,J_y) >0$, $J > (J_x,J_y) $ and that $J_x \neq J_y$. Introduce the Holstein-Primakoff-transformation for the ferromagnet, calculate to quadratic order in boson-operators, and show that the Hamiltonian may be written on the form
\begin{eqnarray}
H = -J N S^2 z + 2 J S z \sum_i ~ a^{\dagger}_i a_i 
- 2 \bar{J} S  \sum_{\langle i,j \rangle} ~  a^{\dagger}_i a_j 
+ \Delta J S \sum_{\langle i,j \rangle} ~  \left[  a^{\dagger}_i a^{\dagger}_j +  a_i a_j \right] \nonumber 
\end{eqnarray}
with $2 \bar{J} \equiv J_x+J_y$ and $2  \Delta J \equiv J_x-J_y$. $N$ is the total number of lattice sites. Note the appearance of the ``anomalous'' terms 
$a^{\dagger}_i a^{\dagger}_j $ and $a_i a_j$ due to 
$J_x \neq J_y$.  Hence, a symplectic Bogoliubov-transformation will be needed to obtain the magnon-spectrum of this system as well. 
\ \\
\ \\
{\bf c)} Introduce first a Fourier-transformed operators like we did in class, and show that 
\begin{eqnarray}
H = -J N S^2 z + \sum_{q}\left\{  \gamma_1(q) \left[ a^{\dagger}_q a_q + a^{\dagger}_{-q} a_{-q}  \right] +  \gamma_2(q) \left[ a_q a_{-q} + a^{\dagger}_{q} a^{\dagger}_{-q}  \right]\right\} \nonumber 
\end{eqnarray}
and give expressions for $\gamma_1(q)$ and $\gamma_2(q)$.
\ \\
\ \\
{\bf d)} Introduce the Bogoliubov-transformation
\begin{eqnarray}
A_{q} & = & u_q a_q + v_q a^{\dagger}_{-q}  \nonumber \\
A^{\dagger}_{-q} & = & u_q a^{\dagger}_{-q} + v_q a_{q}  \nonumber
\end{eqnarray}
and find expressions for $u_q$ and $v_q$ that diagonalize the Hamiltonian. 
\ \\
\ \\
{\bf e)} Explain what the main difference between the squeezing-factors  $u_q$ and $v_q$ in the present case is, and the squeezing factors we found in the isotropic quantum antiferromagnet.  Comment on the presence of quantum-fluctuations in the ground state of the anisotropic quantum ferromagnet, compared to those that are present in the isotropic quantum-antiferromagnet. (Hint: Consider the limit $q \to 0$ of $u_q$ and $v_q$ for both cases
and compare). 
\ \\
\ \\
{\bf f)} Squeezing of magnons has potentially very useful applications. One possible application is in low-dissipation heterostructures where magnons may induce loss of dissipation (electrical resistance) in heterostrucure of magnetic insulators and normal metals.  Electronic devices where electrical dissipation can be eliminated or strongly reduced, is of obvious practical importance. To facilitate such loss or reduction, it turns out that large squeezing is important. Given this, what would you suggest is the most efficient heterostructure to use: i) an NM/FM-structure or an NM/AFM-structure? Here, NM means normal metal, FM is anisotropic quantum ferromagnet, and AFM is isotropic quantum antiferromagnet.    





\end{document}

