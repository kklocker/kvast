
\section*{Problems}
\addcontentsline{toc}{section}{Problems}
% HWP 8_V21

\begin{problem}
	
The creation and destruction operator of a Cooper-pair is given by 
\begin{eqnarray}
	b_{\bf k} & =  & c_{-{\bf k}, \downarrow} c_{{\bf k}, \uparrow} \nonumber \\  
	b^{\dagger}_{\bf k} & =  & c^{\dagger}_{{\bf k}, \uparrow} c^{\dagger}_{-{\bf k}, \downarrow} \nonumber  
\end{eqnarray}
\ \\
\ \\
{\bf a) } Compute the commutators $\left[ b_{\bf k}, b^{\dagger}_{{\bf k}^{\prime}} \right]$  and 
$\left[ b_{\bf k}, b_{{\bf k}^{\prime}} \right]$, and the anti-commutator  $\left\{ b_{\bf k}, b_{{\bf k}^{\prime}} \right\}$. 
Compare your results with what you find if the operators had been boson-operators. 
\ \\
\ \\
{\bf b)} Imagine that you had two different condensed matter systems, one with a relatively strong attraction between electrons, and one with a relatively weak attraction between electrons.  Which one of these two system would you think a description of Cooper-pairs as bosons would be the best approximation?  
\end{problem}
\begin{problem}
	
In Problem 1 we considered creation an destruction operators for a Cooper-pair. Consider now a bosonic {\it pair-fluctuation field} $\phi({\bf q}, \omega)$ which can split into two electrons, and two electrons can recombine into the field $\phi({\bf q}, \omega)$. 
\ \\
\ \\
A Feynman-diagram for such a process is given in the Figure below. 
\ \\
\ \\
\begin{center}
	\begin{tikzpicture}
		\node at (11, 0) {$= \chi_0({\bf q}, \omega)  ~  K({\bf q},\omega)  ~ \chi_0({\bf q}, \omega) $};
		\begin{feynman}
			\vertex (a);
			\vertex[right=2cm of a] (b);
			\vertex[right=4cm of b] (c);
			\vertex[right=2cm of c] (d);
			\diagram*{(a)--[charged boson, edge label = \(q \comma \omega\)] (b),
				(b) -- [fermion, quarter left, edge label= \(q - k \comma \omega - \omega'\)] (c), 
				(b) -- [fermion, quarter right, edge label' = \(k \comma \omega'\)] (c),
				(c) -- [charged boson, edge label = \( q \comma \omega \)] (d)  
			};
		\end{feynman} 
	\end{tikzpicture}
\end{center}
\ \\
\ \\
The wavy line is the Green's function for the free field $\phi({\bf q},\omega)$, denoted $\chi_0({\bf q}, \omega)$.
Denote the bubble-diagram by $K({\bf q},\omega)$. Consider now a Dyson-equation for the Green's function 
$\chi({\bf q}, \omega)$ for the pairing field
\ \\
\ \\
\begin{eqnarray}
	\chi({\bf q}, \omega)^{-1} & = & \chi_0({\bf q}, \omega)^{-1} - K({\bf q},\omega) \nonumber \\
	\chi_0({\bf q}, \omega)^{-1} & = & V \nonumber
\end{eqnarray} 
Here, $V=$ is a constant attractive interaction that works in a thin shell around the Fermi-surface.  
\ \\
\ \\
{\bf a)} Use the Feynman-rules to compute the integral over $\omega^{\prime}$ in the bubble-diagram, taking care to express your answer in terms of $T=0$ Fermi-distribution $\theta(\varepsilon_F-\varepsilon_{\bf k})$ and its complementary  distribution $\theta(\varepsilon_{\bf k} - \varepsilon_F )$, and corresponding quantities with 
${\bf k} \to {\bf k} +{\bf q}$.
\ \\
\ \\
{\bf b)} Generalize this to finite temperatures $T>0$ by replacing the $\theta$-functions by finite-temperature Fermi-distributions and simplify the expression by letting ${\bf q} \to 0$.   
\ \\
\ \\
{\bf c)}$\chi({\bf q}, \omega)$ may be given a physical interpretation as a {\it pair-susceptibility}, i.e. the ability an electron system has to form pairs of electrons, Cooper-pairs. The divergence of this susceptibility indicates an instability of the Fermi-sea of electrons. Find the temperature at which this instability takes place by explicitly computing the remaining ${\bf k}$-sum. (Hint: Consider the case ${\bf q}=0$ and convert the sum over ${\bf k}$ to an energy integral and approximate the single-particle density of states by its value on the Fermi-surface.)
\ \\
\ \\
Note how very differently the particle-particle bubble $K({\bf q},\omega)$ behaves from the particle-hole bubble
$\Pi({\bf q},\omega) $ we considered in class.
\begin{center}
	\begin{tikzpicture}
		\node at (11, 0) {$= D_0({\bf q}, \omega)  ~  \Pi({\bf q},\omega)  ~ D_0({\bf q}, \omega) $};
		\begin{feynman}
			\vertex (a);
			\vertex[right=2cm of a] (b);
			\vertex[right=4cm of b] (c);
			\vertex[right=2cm of c] (d);
			\diagram*{(a)--[charged boson, edge label = \(q \comma \omega\)] (b),
				(b) -- [fermion, quarter left, edge label= \(q + k \comma \omega + \omega'\)] (c), 
				(c) -- [fermion, quarter left, edge label' = \(k \comma \omega'\)] (b),
				(c) -- [charged boson, edge label = \( q \comma \omega \)] (d)  
			};
		\end{feynman} 
	\end{tikzpicture}
\end{center}
There, we found that $\lim_{{\bf q} \to 0} \Pi({\bf q},\omega=0) = - 2 N(\varepsilon_F)$. This is essentially $T$-independent. 
\end{problem}


\begin{problem}
	
	In class, we have introduced new operators $\eta_{\bf k}, \gamma_{\bf k}$ to diagonalize the mean-field BCS-Hamiltonian using transformation coefficients $(u_{\bf k}, v_{\bf k})$ that are real. Let us generalize this to the case where $(u_{\bf k}, v_{\bf k})$  can be complex. As in class, we choose $V_{{\bf k},{\bf k}^{\prime}}$ to be a constant within a thin shell around the Fermi surface. We will use the following transformation
	\[
	\begin{mpmatrix}
		\eta_{ \bf k } \\
		\gamma_{ \bf k }
	\end{mpmatrix} 
	=
	\begin{mpmatrix}
		u_{\bf k} & v_{\bf k} \\
		- v^{*}_{\bf k} & u^{*}_{\bf k}
	\end{mpmatrix} 
	\begin{mpmatrix}
		c_{{\bf k}, \uparrow}\\
		c^{\dagger}_{-{\bf k},\downarrow} 
	\end{mpmatrix} 
	\]
	{\bf a)} Show that preservation of anti-commutation relations gives the constraint 
	$$|u_{\bf k}|^2+|v_{\bf k}|^2=1 $$
	\ \\
	\ \\
	{\bf b)} We next want to diagonalize the problem such that only terms of the form $ \gamma^{\dagger}_{\bf k }  \gamma_{\bf k }$ 
	and  $ \eta^{\dagger}_{\bf k }  \eta_{\bf k }$ appear in the Hamiltonian. Show that the condition for this is given by
	\begin{eqnarray}
		-2 (\varepsilon_{\bf k} - \mu) u_{\bf k} v_{\bf k} & = & u_{\bf k}^2 \Delta - v_{\bf k}^2 \Delta^{*}  \nonumber \\
		-2 (\varepsilon_{\bf k} - \mu) u^{*}_{\bf k}  v^{*}_{\bf k} & = & (u^{*}_{\bf k})^2 \Delta^{*} - (v^{*}_{\bf k})^2 \Delta  \nonumber 
	\end{eqnarray}
	\ \\
	\ \\
	{\bf c)} Show that the coefficient of $\gamma^{\dagger}_{ \bf k }  \gamma_{\bf k }$ is given by
	\begin{eqnarray}
		E_{\bf k} = \left( \varepsilon_{\bf k} - \mu \right) \left( |v_{\bf k}|^2-|u_{\bf k}|^2 \right) 
		+ \Delta v^{*}_{\bf k} u_{\bf k} + \Delta^{*} v_{\bf k} u^{*}_{\bf k} \nonumber
	\end{eqnarray}
	and that the corresponding coefficient  of $\eta^{\dagger}_{ \bf k }  \gamma_{ \bf k }$ is given by  $- E_{\bf k} $.  
	\ \\
	\ \\
	{\bf d)} We next parametrize $u = \cos \chi ~e^{i \phi_u}$, $v =  \sin \chi ~e^{i \phi_v} $, $\Delta = |\Delta| e^{i \phi}$. Use the constraint in ${\bf b}$ to find constraints on the phases $\phi_u,\phi_v,\phi$. Conclude from this that $E_{\bf k}$ is real. 
	\ \\
	\ \\
	{\bf e)} Use the definition of $\Delta$ as given in class to show that in this case, the phase of the gap may be cancelled out of the self-consistent equation for $\Delta$. Hence, in this case, we can consider $\Delta$ to be real and positive without loss of generality.   
\end{problem}
\begin{problem}
	
	In this problem, we will consider a generalization of the BCS-theory to a situation where electrons originating with two energy-bands can participate in superconductivity. Such a situation arises in transition metal elements, where scattering of electrons in $s$- and $d$-orbitals constributes to the resistivity in the normal state. The generalization of the BCS-reduced Hamiltonian to this case is  straightforward. We denote the single-particle excitation energies of the electrons on the $s$- and $d$-bands as $\varepsilon_{{\bf k} s}$ and $\varepsilon_{{\bf k} d}$, respectively.
	\ \\
	\ \\
	The BCS-reduced Hamiltonian for this case is given by
	\begin{eqnarray}
		\Ha&  = &  \sum_{{\bf k},\sigma} \left( \varepsilon_{{\bf k} s} - \mu \right) ~ c^{\dagger}_{{\bf k},\sigma} c_{{\bf k},\sigma}
		+  \sum_{{\bf k},\sigma} \left( \varepsilon_{{\bf k} d} - \mu \right)  ~ d^{\dagger}_{{\bf k},\sigma} d_{{\bf k},\sigma} \nonumber \\
		& - & V_{ss} \sum_{{\bf k},{\bf k}^{\prime}} c^{\dagger}_{{\bf k}^{\prime},\uparrow}c^{\dagger}_{-{\bf k}^{\prime},\downarrow}
		c_{-{\bf k},\downarrow}c_{{\bf k},\uparrow}
		- V_{dd} \sum_{{\bf k},{\bf k}^{\prime}} d^{\dagger}_{{\bf k}^{\prime},\uparrow}d^{\dagger}_{-{\bf k}^{\prime},\downarrow}
		d_{-{\bf k},\downarrow} d_{{\bf k},\uparrow} \nonumber \\
		& - & V_{sd} \sum_{{\bf k},{\bf k}^{\prime}} \left[  d^{\dagger}_{{\bf k}^{\prime},\uparrow}d^{\dagger}_{-{\bf k}^{\prime},\downarrow}
		c_{-{\bf k},\downarrow} c_{{\bf k},\uparrow} 
		+c^{\dagger}_{{\bf k}^{\prime},\uparrow}c^{\dagger}_{-{\bf k}^{\prime},\downarrow}
		d_{-{\bf k},\downarrow} d_{{\bf k},\uparrow} \nonumber
		\right]
	\end{eqnarray}
	Here, $(V_{ss}, V_{dd}, V_{sd})>0$ are attractive interactions operative provided both ${\bf k},{\bf k}^{\prime}$ both are located within a thin shell around the Fermi-surface of the $s$- and $d$-bands. The $c$- and  $d$-operators are fermionic creation- and destruction operators of the $s$ and $d$-bands,  respectively. Note that the Hamiltonian decouples into two independent single-band problems if $V_{sd}=0$.
	\ \\
	\ \\
	The $V_{ss}$- and $V_{dd}$-terms are intra-band scattering processes on the $s$- and $d$-bands, respectively. The $V_{sd}$-term is an inter-band scattering between the $s$- and the $d$-band. 
	\ \\
	\ \\
	The summations of ${\bf k},{\bf k}^{\prime}$
	are close to the Fermi-surface both for the $s$- and the $d$-band, so the regions in ${\bf k}$-space will be quite different in the two cases. Furthermore, we will denote the single-particle densities of states on the Fermi-surface in the $s$- and $d$-bands as $N_s$ and $N_d$, respectively. 
	\ \\
	\ \\
	The Feynman-diagram illustrating the various scattering processes is given below. 
	\vskip 1.0cm
	
	\begin{center}
		
		\tikzset{
			particle/.style={thick,draw=blue, postaction={decorate},
				decoration={markings,mark=at position .5 with {\arrow[blue]{triangle 45}}}},
			scalar/.style={thick,draw=red, postaction={decorate},
				decoration={markings,mark=at position .5 with {\arrow[red]{triangle 45}}}},  
			gluon/.style={decorate, draw=black,
				decoration={coil,aspect=0}}
		}
		\tikzset{
			particle/.style={thick,draw=blue, postaction={decorate},
				decoration={markings,mark=at position .5 with {\arrow[blue]{stealth}}}},
			scalar/.style={thick,draw=red, postaction={decorate},
				decoration={markings,mark=at position .5 with {\arrow[red]{stealth}}}},
			gluon/.style={decorate, draw=black, decoration={snake=coil}},
			photon/.style={decorate, decoration={snake}, draw=black},
		}
		
		\begin{tikzpicture}[node distance=1.5cm and 1.5cm]
			\coordinate[label=below:$\alpha  \comma  {\bf k}   \comma \uparrow$] (e1);
			\coordinate[above right=of e1] (aux1);
			\coordinate[above left=of aux1,label=above:$\beta \comma  {\bf k}^{\prime}  \comma \uparrow$] (e2);
			\coordinate[right=2.0cm of aux1] (aux2);
			\coordinate[below right=of aux2,label=below:$\alpha \comma - {\bf k}   \comma \downarrow$] (e3);
			\coordinate[above right =of aux2,label=above:$\beta \comma  -{\bf k}^{\prime}  \comma \downarrow$] (e4);
			
			\draw[scalar] (e1) -- (aux1);
			\draw[particle] (aux1) -- (e2);
			\draw[photon] (aux1) --node[label=above:$V_{\alpha \beta}$]{} (aux2);
			\draw[scalar] (e3) -- (aux2);
			\draw[particle] (aux2) --  (e4);
		\end{tikzpicture}
	\end{center}
	\ \\
	\ \\
	{\bf a)} We next perform a mean-field approximation along the same lines that we did in class for the single-band case. Introduce mean-field expectation values 
	\begin{eqnarray}
		b_{{\bf k},s} & = & \langle c_{-{\bf k},\downarrow}c_{{\bf k},\uparrow} \rangle \nonumber \\
		b_{{\bf k},d} & = & \langle d_{-{\bf k},\downarrow}d_{{\bf k},\uparrow} \rangle \nonumber 
	\end{eqnarray}
	and write 
	\begin{eqnarray}
		c_{-{\bf k},\downarrow}c_{{\bf k},\uparrow} & = & b_{{\bf k},s} + \delta b_{{\bf k},s}  \nonumber \\
		d_{-{\bf k},\downarrow}d_{{\bf k},\uparrow} & = & b_{{\bf k},d} + \delta b_{{\bf k},d}  \nonumber
	\end{eqnarray}
	and discard terms ${\cal O} (\delta b_{{\bf k},s})^2$ and ${\cal O} (\delta b_{{\bf k},s})^2$. Show that the mean-field Hamiltonian may be written on form
	\begin{eqnarray}
		\Ha&  = &  \sum_{{\bf k},\sigma} \left( \varepsilon_{{\bf k} s} - \mu \right) ~ c^{\dagger}_{{\bf k},\sigma} c_{{\bf k},\sigma}
		+  \sum_{{\bf k},\sigma} \left( \varepsilon_{{\bf k} d} - \mu \right)  ~ d^{\dagger}_{{\bf k},\sigma} d_{{\bf k},\sigma} \nonumber \\
		& - &  V_{ss} \sum_{{\bf k},{\bf k}^{\prime}} 
		\left[ b^{\dagger}_{{\bf k}^{\prime},s} ~ c_{-{\bf k},\downarrow}c_{{\bf k},\uparrow} + h.c. \right] 
		- V_{dd} \sum_{{\bf k},{\bf k}^{\prime}} 
		\left[ b^{\dagger}_{{\bf k}^{\prime},d} ~ d_{-{\bf k},\downarrow} d_{{\bf k},\uparrow} + h.c. \right] \nonumber \\
		& - & V_{sd} \sum_{{\bf k},{\bf k}^{\prime}} ~
		\left[ b^{\dagger}_{{\bf k}^{\prime},d} ~c_{-{\bf k},\downarrow} c_{{\bf k},\uparrow} + h.c. 
		+
		b^{\dagger}_{{\bf k}^{\prime},s} ~d_{-{\bf k},\downarrow} d_{{\bf k},\uparrow}  + h.c. \right] \nonumber \\
		& + &
		V_{ss}\sum_{{\bf k},{\bf k}^{\prime}} ~b^{\dagger}_{{\bf k}^{\prime},s}~b_{{\bf k},s}
		+
		V_{dd}\sum_{{\bf k},{\bf k}^{\prime}} ~b^{\dagger}_{{\bf k}^{\prime},d}~b_{{\bf k},d}
		+
		V_{sd}\sum_{{\bf k},{\bf k}^{\prime}}
		\left[ b^{\dagger}_{{\bf k}^{\prime},d}~b_{{\bf k},s} + b^{\dagger}_{{\bf k}^{\prime},s}~b_{{\bf k},d}  \right]  \nonumber
	\end{eqnarray}
	\ \\
	\ \\
	{\bf b)} Introduce the two quantities $\Delta_1$ and $\Delta_2$ (to be dermined selfconsistently)
	\begin{eqnarray}
		\Delta_1 & \equiv & V_{ss} \sum_{\bf k}b_{{\bf k},s} + V_{sd} \sum_{\bf k} b_{{\bf k},d} \nonumber \\
		\Delta_2 & \equiv & V_{sd} \sum_{\bf k}b_{{\bf k},s} + V_{dd} \sum_{\bf k} b_{{\bf k},d} \nonumber
	\end{eqnarray}
	Express the mean-field Hamiltonian in terms of $\Delta_1$ and $\Delta_2$ instead of $b_{{\bf k},s}$ and $b_{{\bf k},d}$. 
	Try to give  a physical interpretation of  $\Delta_1$ and $\Delta_2$. (Hint: Try to infer what $\Delta_1$ and $\Delta_2$ mean based on how they appear in the Hamiltonian). 
	\ \\
	\ \\
	{\bf c)} To diagonalize the mean-field problem, we will introduce new fermionic operators $e_{{\bf k},\sigma}$ and $f_{{\bf k},\sigma}$ as follows
	\begin{eqnarray}
		c_{{\bf k},\uparrow} & = & 
		\cos(\theta_{\bf k}) e_{{\bf k},\uparrow} +  \sin(\theta_{\bf k}) e^{\dagger}_{-{\bf k},\downarrow}  \nonumber \\
		c_{{\bf k},\downarrow} & = & 
		\cos(\theta_{\bf k}) e_{{\bf k},\downarrow} - \sin(\theta_{\bf k}) e^{\dagger}_{-{\bf k},\uparrow} \nonumber \\
		d_{{\bf k},\uparrow} & = & 
		\cos(\phi_{\bf k}) f_{{\bf k},\uparrow} +  \sin(\phi_{\bf k}) f^{\dagger}_{-{\bf k},\downarrow}  \nonumber \\
		d_{{\bf k},\downarrow} & = & 
		\cos(\phi_{\bf k}) f_{{\bf k},\downarrow} - \sin(\phi_{\bf k}) f^{\dagger}_{-{\bf k},\uparrow} \nonumber 
	\end{eqnarray}
	The parameters $\theta_{\bf k}$ and $\phi_{\bf k}$ are determined by substituting these expressions into $\Ha$ and equating
	the coefficients of terms $e^{\dagger}e^{\dagger}  $ and $e e$ to zero, and likewise for the $f$-operators. Show that this procedure yields the following equations determining  $\theta_{\bf k}$ and $\phi_{\bf k}$
	\begin{eqnarray}
		\varepsilon_{{\bf k} s}  \sin(2 \theta_{\bf k}) +  D_s \cos(2 \theta_{\bf k})  & = & 0 \nonumber \\
		\varepsilon_{{\bf k} d}  \sin(2 \phi_{\bf k}) +    D_d \cos(2 \phi_{\bf k})  & = & 0 \nonumber 
	\end{eqnarray}
	and give expressions for $D_s$ and $D_d$. \\
	Find the excitation energies of the Bogoliubov quasiparticles $(e,f)$ defined above.
	\ \\
	\ \\
	{\bf d)} Find an expression for the free energy of the system.
	\ \\
	\ \\
	{\bf e)} Minimize the free energy w.r.t the quantities $\Delta_1$ and $\Delta_2$ and show that the coupled equations for these two quantities may be written
	\begin{eqnarray}
		\Delta_{\alpha} & = & \sum_{\beta=1}^2 V_{\alpha \beta} \Delta_{\beta} ~ \chi_{\beta};  \alpha = (1,2)  \nonumber \\
		\chi_{\beta} & \equiv & \sum_{\bf k} ~ \frac{1}{2 E_{\beta \bf k} } \tanh \left( \frac{\beta E_{\beta \bf k}}{2} \right) \nonumber
	\end{eqnarray}
	\ \\
	\ \\
	{\bf f)} With the real transformation coefficients we have used so far, $\Delta_\alpha$ will be real. Imagine that we had used complex transformation coefficients instead, like in Problem 1. We would then get the same equations for $\Delta_{\alpha}$ as above, but  with the possibility of complex gaps, $\Delta_{\alpha} = |\Delta_{\alpha}| e^{i \phi_\alpha}$. We may thus cancel out an overall phase from the gap-equations, and be left with {\it one and only one} relative phase $\phi_{12}= \phi_1-\phi_2$. Show, using the gap-equations, that $\phi_{12} = (0,\pi) \mod(2 \pi)$. Therefore, the gaps again may both be taken to be real, but they may have opposite signs. What is the criterion for $\Delta_1$ and $\Delta_2$ having equal or opposite signs?
	\ \\
	\ \\
	Note: This sort of simplification in general is not possible when one has more than two bands involved in the superconductivity. This leads to qualitatively new effects. Important examples of $N$-band superconductors with $N \geq 3$ are the iron-pnictide high-$T_c$ superconductors currently of considerable interest.  
	\ \\
	\ \\
	{\bf g)} Find an expression for the critical temperature of the system using the same technique that we used in class for the single-band case.  
\end{problem}

