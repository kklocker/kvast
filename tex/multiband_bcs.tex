
\section{Multi-band superconductors}
% Ln 12

In many new materials, we may have a situation where several energy-bands can cross the Fermi-level. In such a situation, we will have multiple Fermi-surfaces that can participate in superconductivity. A prominent example of this is superconductivity in the \emph{iron-particles}, where at least three bands are involved in superconductivity. These materials are potentially important and of great interest because they  are high-$T_C$ superconductors, and because they feature interesting new physics precisely because multiple bands are involved in superconductivity. Let us try to set up a theory of such superconductors within the framework we have used up until now. We introduce an index $\alpha$ for an energy band $\ep_{k\alpha}$.
A BCS-reduced Hamiltonian is
\begin{tcolorbox}
	\begin{equation}		
		\Ha =
		 \sum_{k,\alpha, \sigma}(\ep_{k\alpha}-\mu)\cd_{k\alpha\sigma}c_{k\alpha\sigma}
		+ \sum_{\substack{k,k' \\ \alpha,\beta}}V_{kk'}^{\alpha\beta}P_{k\alpha}^{\dg}P_{k'\beta}
	\end{equation}
\end{tcolorbox}
where 
\begin{equation}
	P_{k\alpha}^{\dg} \equiv \cd_{k\alpha\uparrow}\cd_{-k\alpha\downarrow}.
\end{equation}
$V_{kk'}^{\alpha\beta}$ scatters a pair of electrons at $k,-k$ in band $\alpha$ to $(k', -k')$ in band $\beta$.
We now perform a mean-field approximation exactly as in the one-band case.
\begin{equation}
	P_{k\alpha}^{\dg} = b_{k\alpha}^{\dg} + \delta P_{k\alpha}^{\dg},
\end{equation}
and ignore $(\delta P_{k\alpha}^{\dg})^2$-terms.
\emph{This gives:}
\begin{equation}
\begin{aligned}
	\Ha &= \sum_{k,\alpha, \sigma}(\ep_{k\alpha}-\mu)\cd_{k\alpha\sigma}c_{k\alpha\sigma} \\
	& + \sum_{\substack{k,k' \\ \alpha,\beta}}V_{kk'}^{\alpha\beta}\left [ b_{k\alpha}^{\dg}\cd_{-k'\beta\downarrow}c_{k'\beta\uparrow}  +  b_{k'\beta}\cd_{k\alpha\uparrow}\cd_{-k\alpha\downarrow} - b_{k\alpha}^{\dg}b_{k'\beta}\right ],
\end{aligned}
\end{equation}
where
\begin{align}
	b_{k\alpha}^{\dg}&\equiv \ev*{\cd_{k\alpha\uparrow}\cd_{-k\alpha\downarrow}} \\ 
	b_{k\beta} &\equiv \ev*{c_{-k\beta\downarrow}c_{k\beta\uparrow}}.
\end{align}
Now define
\begin{align}
	\Delta_{k\alpha} &= -\sum_{k'}V_{kk'}^{\alpha\beta}b_{k'\beta} \\
	\Delta_{k'\beta}^{\dg} &= -\sum_{k}V_{kk'}^{\alpha\beta}b_{k\alpha}^{\dg}.
\end{align}\todo{Skal det summeres over $\alpha$ her, eller ligger denne summasjonen implisitt i gjentagende indekser?}
\emph{Then}, at mean-field level: 
\begin{tcolorbox}
	\begin{equation}
		\begin{aligned}
			\Ha &= \sum_{k,\alpha}\Delta_{k\alpha}b_{k\alpha}^{\dg} \\
			&+\sum_{k,\alpha, \sigma}(\ep_{k\alpha}-\mu)\cd_{k\alpha\sigma}c_{k\alpha\sigma} \\
			&- \sum_{k,\alpha}\left [\Delta_{k\alpha}^{\dg}P_{k\alpha} + \Delta_{k\alpha}P_{k\alpha}^{\dg}\right ]
		\end{aligned}
	\end{equation}
\end{tcolorbox}

Note how $\Delta_{k\alpha}, \Delta_{k\alpha}^{\dg}$ act as source- and sink-fields for Cooper-pairs in band $\alpha$. However,  $\Delta_{k\alpha}, \Delta_{k\alpha}^{\dg}$ have contributions from all bands, since $V_{kk'}^{\alpha\beta}$ provides inter-band ($\alpha = \beta$) as well as inter-band ($\alpha\ne\beta$) scattering. Note also that $\Ha$ is now ``diagonalized '' in band-indices, and we may diagonalize the problem exactly as in the one-band case. 
\emph{Introduce operators:}
\begin{tcolorbox}
	\begin{equation}
		\begin{aligned}
			\eta_{k\alpha}^{\dg} &= u_{k\alpha}\cd_{k\alpha\uparrow} + v_{k\alpha}c_{-k\alpha\downarrow} \\
			\gamma_{k\alpha}^{\dg} &= u_{k\alpha}c_{-k\alpha\downarrow} -v_{k\alpha} \cd_{k\alpha\uparrow}
		\end{aligned}
	\end{equation}
\end{tcolorbox}
For each band $\alpha$, the calculations are now exactly as for the one-band case. Thus, we obtain
\begin{equation}
	\begin{aligned}
	\Ha &= \sum_{k,\alpha}\left [(\ep_{k\alpha}-\mu) + \Delta_{k\alpha}b_{k\alpha}^{\dg}\right ] \\& + 
	\sum_{k,\alpha}E_{k\alpha}\left (\gamma_{k\alpha}^{\dg}\gamma_{k\alpha} - \eta_{k\alpha}^{\dg}\eta_{k\alpha}\right ).
	\end{aligned}
\end{equation}
From this, we obtain the free energy
\begin{equation}
	\begin{aligned}
		F &= \sum_{k,\alpha}\left [(\ep_{k\alpha}-\mu)+ \Delta_{k\alpha}b_{k\alpha}^{\dg}\right ]  \\
		&-\sum_{k,\alpha}\left [\ln(1+\e^{-\beta E_{k\alpha}}) +\ln(1 + \e^{\beta E_{k\alpha}})\right ]
	\end{aligned}
\end{equation}

$V_{kk'}^{\alpha\beta} = V_{\alpha\beta}$, $k,k'$ in thin shells.
\begin{equation}
	\Delta_{k\alpha}^{\dg} = -\sum_{k',\beta}V_{kk'}^{\alpha\beta}b_{k'\beta}^{\dg} = \Delta_{\alpha}^{\dg}.
\end{equation}
Define 
\begin{equation}
	b_{\beta}^{\dg} = \sum_{k}b_{k\beta}^{\dg},
\end{equation}
\begin{align}
	\Delta_{\alpha}^{\dg} &= -\sum_{\beta}V_{\alpha\beta}b_{\beta}^{\dg} \\
	\bm{\Delta}^{\dg} &= -{V}\cdot \bm{b}^{\dg} \implies \bm{b}^{\dg} = -{V}^{-1}\cdot\bm{\Delta}^{\dg}.
\end{align}
\emph{Thus, we have:}
\begin{align}
	\sum_{k,\alpha}\Delta_{k\alpha}b_{k\alpha}^{\dg}&= \bm{\Delta}\cdot\bm{b}^{\dg}\nonumber \\
	&= - \bm{\Delta}\cdot ({V}^{-1})\cdot \bm{\Delta}^{\dg} \nonumber \\
	&= -\sum_{\alpha,\beta}\Delta_{\alpha}(V^{-1})_{\alpha\beta}\Delta_{\beta}^{\dg}.
\end{align}
\begin{equation}
	V_{\alpha\beta} = V_{\beta\alpha} \implies (V^{-1})_{\alpha\beta})=(V^{-1})_{\beta\alpha}.
\end{equation}
\begin{equation}
	E_0 = \sum_{k,\alpha}(\ep_{k\alpha}-\mu)- \sum_{\alpha, \beta}\Delta_{\alpha}(V^{-1})_{\alpha\beta}\Delta_{\beta}^{\dg}.
\end{equation}
We now find an equation for $\Delta_\alpha$ from $\pdv{F}{\Delta_{\alpha}} = 0$.
\begin{tcolorbox}
	\begin{equation}
		\pdv{E_0}{\Delta_{\alpha}} - \sum_{k}\frac{\Delta_{\alpha}^{\dg}}{2E_{k\alpha}}\tanh(\frac{\beta E_{k\alpha}}{2}) = 0.
	\end{equation}
\end{tcolorbox}

\begin{align}
	-\sum_{\beta}&(V^{-1})_{\alpha\beta}\Delta_{\beta}^{\dg}- \underbrace{\Delta_{\alpha}^{\dg}\chi_{\alpha}}_{\equiv\tilde{\Delta}_{\alpha}^{\dg}} = 0 \\
	\chi_{\alpha} &\equiv \sum_{k}\frac{1}{2E_{k\alpha}}\tanh(\frac{\beta E_{k\alpha}}{2})
\end{align}
On matrix-vector form, we have
\begin{align}
	-{V}^{-1}\cdot &\bm{\Delta}^{\dg} = \tilde{\bm{\Delta}}^{\dg}\\
	\bm{\Delta}^{\dg} &= -V\cdot \tilde{\bm{\Delta}}^{\dg},
\end{align}
or equivalently
\begin{equation}
	\Delta_{\alpha}^{\dg} = -\sum_{k,\beta}V_{\alpha\beta}\frac{\Delta_{\beta}^{\dg}}{2E_{k\beta}}\tanh(\frac{\beta E_{k\beta}}{2}).
\end{equation}
\begin{equation}
	\Delta_{\alpha} = |\Delta_{\alpha}|\e^{i\theta_{\alpha}}
\end{equation}
Notice how phases $\theta_{\alpha}$ of $\Delta_{\alpha}$ do not cancel in the multiband-case, unlike the one-band case.


This will lead to new effects, namely as follows:
Consider again the term in $F$ 
\begin{equation}
\begin{aligned}
	&-\sum_{\alpha,\beta}\Delta_{\alpha}(V^{-1})_{\alpha\beta}\Delta_{\beta}^{\dg} \\
	&= -\sum_{\alpha}|\Delta_{\alpha}|^2(V^{-1})_{\alpha\beta} - 2\sum_{\alpha<\beta}|\Delta_{\alpha}||\Delta_{\beta}|(V^{-1})_{\alpha\beta}\cos(\theta_{\alpha}-\theta_{\beta}).
\end{aligned}
\end{equation}
In the second term, $\alpha\ne\beta$. If $(V^{-1})_{\alpha\beta}>0$, then this term is minimized by $\theta_{\alpha}-\theta_{\beta} = 0$ ($\cos(\theta_{\alpha}-\theta_{\beta})=1$).
If some $(V^{-1})_{\alpha\beta}<0$, the phases $\{\theta_{\alpha}\}$ of the gaps may be frustrated. Some $\theta_{\alpha}-\theta_{\beta}$ would like to be $0$, while others would like to be $\pi$. In general, for $N\ge3$, these constraints are not compatible, and the system is forced to choose a compromise. A situation that could arise could be the following ($N=3$)
\begin{center}
	\begin{tikzpicture}[scale=2]
		\draw[thick, ->] (0,0) to (0,1) node[anchor=south] {$\theta_1$};
		\draw[thick, ->] (0,0) to (-0.866,-0.5) node[anchor = east] {$\theta_2$};
		\draw[thick, ->] (0,0) to (0.866,-0.5)node[anchor = west] {$\theta_3$};
	\end{tikzpicture}
\end{center}
Here, we may rotate the triad like a ``rigid'' body, this is a $U(1)$ symmetry. Swapping $\theta_2 \leftrightarrow \theta_3$; $\theta_1 \rightarrow\theta_1$ is like an ``Ising''-transformation.
In this case, the system may break a $U(1)\cross Z_2$-symmetry in separate phase-transitions, leading to novel phase-transitions within the superconducting state. 
