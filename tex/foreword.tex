\section*{Foreword}
\addcontentsline{toc}{section}{Foreword to the digitalized lecture notes}

Digitialized lecture notes for the course ``TFY4210 - Quantum Theory of Many-Particle Systems'' held by Prof. Asle Sudbø spring 2020. These notes follow that of the hand written lecture notes, which are based upon the lecture notes for the course ``FY8302 - Quantum Theory of Solids'', originally written in 1996. \\
Course website: \href{https://www.ntnu.edu/studies/courses/TFY4210}{https://www.ntnu.edu/studies/courses/TFY4210}



\subsection*{Initial contributors}

The transcription of lecture notes was started in the spring of 2020. Several people has contributed to the initial transcription. 
\begin{itemize} 
	% 	\item Karl Kristian Lockert
	\item \textbf{Øyvind Muldal Taraldsen} wrote the indtroductory chapters on many-particle system, up until the section on atomic orbital basis. (Lecture notes week 1-2 and start of 3.)
	\item \textbf{Ola Lajord} transcribed the section on many-boson systems, and much of the section on the transformation from Hubbard model to antiferromagnet. (Lecture notes week 4 and start of week 5).
	\item \textbf{Snorre Bergan} did the initial transcription of magnon mediated superconductivity and the Cooper problem. (Most of lecture notes week 9)
	\item \textbf{Niels Henrik Aase} took care of the Ginzburg-Landau theory of superconductivity, including figures. (Lecture notes week 11)
	\item \textbf{Karl Kristian Ladegård Lockert} transcribed most of the remaining lecture notes, and many figures in sections transcribed by other contributors. (Lecture notes week 6-8, start of 9, and 10)
	% 	Put your name here
\end{itemize}

\subsection*{Completion}
The completion of transcription is done during the summer of 2021 by Karl Kristian.

