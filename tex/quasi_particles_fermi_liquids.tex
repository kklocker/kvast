\clearpage
\section{Quasi-particles in interacting electron-systems. Fermi-liquids.}

\subsection{Fermi-liquids}

In a non-interacting electron-system, we have seen that the Green's function (also often called the \textbf{propagator}) has the form 
\begin{equation} 
G_0(k,\omega) = \frac{1}{\omega-\ep_k + \delta_k},
\end{equation}
where we have defined 
\begin{equation} 
\delta_k = \delta\sign(\ep_k-\ep_F).
\end{equation}
The simple poles mean that the system has well-defined and long-lived single-particle excitations, since $\delta = 0^+$. The excitation energy is determined by $\omega-\ep_k$, and the lifetime, $\tau$, of the excitation is given by 
\begin{equation} 
\tau_k =\frac{1}{\delta} \rightarrow \infty ; \quad \delta\rightarrow0^+.
\end{equation}
In the interacting case, we have Dyson's equation
\begin{equation} 
G^{-1}(k,\omega) = G_0^{-1}(k,\omega) - \Sigma(k,\omega),
\end{equation}
or equivalently
\begin{align} 
G(k,\omega) &= \frac{1}{\omega - \ep_k - \Sigma(k,\omega)} \\
G_0(k,\omega) &= \frac{1}{\omega - \ep_k+\frac{i}{\tau_k}}. 
\end{align}
Note that for $G_0(k,\omega)$, the \textbf{residue} at $\omega = \ep_k-i\delta_k$ is $\Res[G_0] = 1$. 
We will now write $G(k, \omega)$ on a similar form 
\begin{equation} 
G(k,\omega) = \frac{z_k}{\omega-\tilde{\ep}_k  + \frac{i}{\tau_k}}
\end{equation}
and find expressions for $z_k, \tilde{\ep}_k, \frac{1}{\tau_k}$ in terms of $\Sigma(k,\omega)$. 
\begin{itemize}
	\item $z_k$: \textbf{Quasiparticle residue}. Physically: How much of the original electron in the non-interacting case remains in the interacting case?
	\item $\tau_k$: \textbf{Quasiparticle lifetime}. 
	\item $\tilde{\ep}_k$: New excitation spectrum of the interacting system.
\end{itemize}
Let us define more precisely what we mean by a \textbf{quasi-particle}. In the non-interacting case, we have well-defined single-particle excitations specified by a set of quantum numbers $(k, \sigma)$, and with an excitation energy $ \ep_k $. For instance, we could have 
\begin{equation} 
\ep_k = \frac{\hbar^2k^2}{2m}
\end{equation}
where $m$ would be the mass of the electron. As we turn on interactions, things will change. The concept of a quasiparticle now assures that there is a \textbf{one-to-one} correspondence between the quantum numbers of the non-interacting case, and the quantum numbers of the interacting case. 
\todo{Sett inn figur}.

The dots represent quantum states. Thus, we may follow the state of an electron into a state with a new set of quantum numbers, and vice versa, as the interactions are turned on and off. The quantum states on the right are quasi-particle states. The renormalization $\ep_k  \rightarrow\tilde{\ep}_k$ could for instance be of the form 
\begin{equation} 
\ep_k = \frac{\hbar^2k^2}{2m} \rightarrow \tilde{\ep}_k = \frac{\hbar^2k^2}{2m^*},
\end{equation}
where $m^*$ is some effective quasiparticle mass different from the mass of the electron. As long as $z_k > 0$, we may talk about \textbf{quasiparticles}. In particular, the crucial issue is what $z_k$ is on the Fermi-surface $k=k_F$, i.e. $z_{k_F}$. So we will focus our attention on what is going on in the vicinity of the Fermi-surface. 
The self-energy is given by
\begin{equation} 
\Sigma(k, \omega) = \Sigma_{\text{Re}}(k,\omega) + i\Sigma_{\text{Im}}(k, \omega).
\end{equation}
We assume that the imaginary part is much less than the real part
\begin{equation} 
\frac{|\Sigma_{\text{Im}}|}{|\Sigma_{\text{Re}}|} \ll 1.
\end{equation}
The quasi-particle poles are found from the zeroes in $G^{-1}(k,\omega)$, i.e.
\begin{tcolorbox}
	\begin{equation} 
		\omega-\ep_k-\Sigma_{\text{Re}} -i\Sigma_{\text{Im}}  = 0	
	\end{equation}
\end{tcolorbox}
To a first approximation, ignore $\Sigma_{\text{Im}}$.
\begin{equation} 
\tikzmark{node1}{\omega}-\ep_k - \Sigma_{\text{Re}}(k,\tikzmark{node2}{\omega}) = 0 = \omega-\tilde{\ep}_k.
\end{equation}
\begin{tikzpicture}[overlay, remember picture,node distance =1.5cm]
\draw[<->,black] (node1) .. controls +(down:1cm) and +(right:0cm) .. node[above] {{\scriptsize NB!}} (node2.south);
\end{tikzpicture}
This is a complicated equation, due to the frequency-dependence of the self-energy. 
\begin{equation} 
\omega = \tilde{\ep}_k = \ep + \Sigma_{\text{Re}}(k, \omega)  = \ep_k + \Sigma_{\text{Re}}(k, \tilde{\ep}_k)
\end{equation}
If the renormalization of $\ep_k \rightarrow \tilde{\ep}_k$ is weak, we may expand the self energy in $\omega$ around $\tilde{\ep}_k$. Anticipating that $\omega$ will change away from $\tilde{\ep}_k$ when $\Sigma_{\text{Im}}$ is introduced:
\begin{equation} 
\Sigma_{\text{Re}}(k,\omega) = \Sigma_{\text{Re}}(k,\tilde{\ep}_k) + (\omega-\tilde{\ep}_k)\underbrace{\left.\pdv{\Sigma_{\text{Re}}}{\omega}\right|_{\omega =  \tilde{\ep}_k}}_{\equiv \Sigma_{\text{Re}}'} + \dots
\end{equation}
Set $\omega = \tilde{\ep}_k + \omega_1$, where $\tilde{\ep}_k$ is a solution to 
\begin{equation}
\label{eq:tilde_e} 
\tilde{\ep}_k = \ep_k + \Sigma_{\text{Re}}(k, \tilde{\ep}_k)
\end{equation}
\begin{align*} 
\omega-  \ep_k - \Sigma_{\text{Re}}(k, \omega) &- i\Sigma_{\text{Im}}(k, \omega) = 0 \implies \\
\tilde{\ep}_k + \omega_1 - \ep_k - \Sigma_{\text{Re}}(k, \tilde{\ep}_k) & - \omega_1\Sigma_{\text{Re}}' - i\Sigma_{\text{Im}}(k, \tilde{\ep}_k) = 0
\end{align*}
Using \cref{eq:tilde_e}, we have
\begin{align*} 
\omega_1 - \omega_!\Sigma_{\text{Re}}' &- i\Sigma_{\text{Im}}(k, \tilde{\ep}_k) = 0 \\
\omega_1 &= i\frac{\Sigma_{\text{Im}}(k, \tilde{\ep}_k)}{1-\Sigma_{\text{Re}}'} \\
\omega = \tilde{\ep}_k-\frac{i}{\tau_k} &= \ep_k + \omega_1 \implies \\
\frac{1}{\tau_k} &= -\frac{\Sigma_{\text{Im}}(k, \tilde{\ep}_k)}{1-\Sigma_{\text{Re}}'}
\end{align*}
This is the quasi-particle lifetime expressed in terms of the self-energy $\Sigma$.
Let us also investigate the quasi-particle residue $z_k$.
\begin{align*} 
G(k, \omega) &= \frac{1}{\omega-\ep_k-\Sigma_{\text{Re}}(k, \omega) - i\Sigma_{\text{Im}}(ki, \omega)} \\
&= \frac{1}{\omega-\tilde{\ep}_k -(\omega-\tilde{\ep}_k\Sigma_{\text{Re}}') -i\left(\frac{1-\Sigma_{\text{Re}}'}{\tau_k}\right)} \\
&= \frac{\left(1-\Sigma_{\text{Re}}'\right)^{-1}}{\omega-\tilde{\ep}_k + \frac{i}{\tau_k}} = \frac{z_k}{\omega - \tilde{\ep}_k + \frac{i}{\tau_k}}.
\end{align*}

\begin{equation} 
z_k = \frac{1}{1-\Sigma_{\text{Re}}'}.
\end{equation}
\begin{equation} 
\Sigma = \begin{gathered}
\feynmandiagram[horizontal=a to b]{a--[fermion] b --[half right, boson] a};
\end{gathered} \implies\Sigma_{\text{Re}}' <0 \implies z_k <1.
\end{equation}
An interacting fermion-system with $z_{k_F}>0$ is called a Fermi-liquid. 

At $T = 0$, the momentum distribution in the non-interacting case follow $n_k = 1-\theta(\ep_k - \ep_F) = \theta(\ep_F-\ep_k)$. This is shown, together with the momentum-distribution for a Fermi-liquid in \cref{fermi_liquid_occupation}
\begin{figure}
	\centering
	\begin{tikzpicture}
\begin{axis}[
	%ticks = none,
	ytick = {1},
	yticklabels = {,,},
	xtick = {0},
	xticklabels = {$k_F$},
	xlabel = {\large $k$},
	ylabel = {\large $n_k$},
	x label style={at={(axis description cs:1,0.1)}, anchor = west},
	y label style={at={(axis description cs:0.15,1)},rotate=-90,anchor=south},
	ymax = 1.7,	
	axis lines = left]


\draw[dashed, red] (axis cs:-5,1) --  (axis cs:0,1);
\draw[dashed, red] (axis cs:0,1) --  (axis cs:0,-0.4);

\addplot[thick,samples=100,black, domain =-5:0] {(1/((exp((x)))+1))};
\addplot[thick,samples=100,black, domain =0:5] {(1/((exp((x)))+1)) - 0.4};
\draw[very thick, black] (axis cs:0,0.5) -- node[anchor= west]{\large $z_k$}  (axis cs:0,0.1);



\end{axis}
\end{tikzpicture}
	\caption{Momentum distribution for a non-interacting system (red dashed line) and a Fermi liquid.}
	\label{fermi_liquid_occupation}
\end{figure}

As long as a discontinuity in $n_k$ exists on the Fermi-surface, we have remnants of \textbf{well defined} single-particle excitations in a one-to-one correspondence with the single-particle states of the non-interacting case
\begin{tcolorbox}
	A Fermi-liquid has $z_{k_F} > 0 $ at $T = 0$.
\end{tcolorbox} 
Note that this statement is a $T=0$-statement, since only at $T =0$ is there a discontinuity in $n_k$ at $k = k_F$
At non-zero $T$, $ 1-\theta(\ep_k - \ep_F) = \theta(\ep_F-\ep_k) $ is replaced by the Fermi-distribution
\begin{equation}
n_k = \frac{1}{\e^{\beta(\ep_k-\mu)}} ; \quad \beta = \frac{1}{k_B T}
\end{equation}
which is an analytic function of $k$ at $k=k_F$.
There are cases where Fermi-liquids are destroyed. A well understood example of this is the on-dimensional interacting electron gas.
In one-dimension, any amount of interacting between electrons destroys $z_{k_F}$. \textbf{For example, the 1-d Hubbard model is not a Fermi-liquid for any $U>0$}. Computing the momentum distribution of the 1-d Hubbard model is difficult, but the result is 
\begin{center}
\begin{tikzpicture}
\begin{axis}[
%ticks = none,
ytick = {1},
yticklabels = {,,},
xtick = {0},
xticklabels = {$k_F$},
xlabel = {\large $k$},
ylabel = {\large $n_k$},
x label style={at={(axis description cs:1,0.1)}, anchor = west},
y label style={at={(axis description cs:0.15,1)},rotate=-90,anchor=south},
ymax = 1.7,	
axis lines = left]


\draw[dashed, red] (axis cs:-5,1) --  (axis cs:0,1);
\draw[dashed, red] (axis cs:0,1) --  (axis cs:0,-0.4);

\addplot[thick,samples=100,black, domain =-5:5] {(1/((exp((2*x)))+1))};
%\addplot[thick,samples=100,black, domain =0:5] {(1/((exp((x)))+1)) - 0.4};
%\draw[very thick, black] (axis cs:0,0.5) -- node[anchor= west]{\large $z_k$}  (axis cs:0,0.1);
\end{axis}
\end{tikzpicture}
\end{center}
not Fermi-liquid. $z_{k_F} = 0$. 
\begin{equation}
n_k=\frac{1}{2}\left(1-\sign\left(k-k_{F}\right)\left|k-k_{F}\right|^{1 / \delta}\right).
\end{equation}
Note, however,m that $n_k$ still is non-analytic at $k=k_F$. Thus, we still have a well-defined Fermi-surface!
In this case, the long-lived low-energy excitations around the Fermi-surface is not in a ne-to-one correspondence with the quantum states of the non-interacting case. This quantum liquid is called a Luttinger-liquid and its single-electron Green's function does not have simple poles, but rather branch-cuts. 
In three dimensions, however, the Fermi-liquid us extremely robust to Coulomb-interactions among electrons. 


\subsection{Screening of the Coulomb-interaction}

\begin{equation}\label{eq:coloumb_int}
V(r) = \frac{e^2}{4\pi\epsilon_0}\frac{1}{r},
\end{equation}
unscreened Coulomb interaction. Fourier-transform
\begin{equation}
\tilde{V}(q) = \frac{1}{V_0}\int\dd{r}\e^{iq\cdot r}
\end{equation}
where $V_0$ is the volume of the system.
Performing the Fourier-transform, we have

\begin{equation}
\tilde{V} = \frac{1}{V_0\epsilon_0}\frac{e^2}{q^2} = \frac{K}{q^2}\quad;\quad K=\frac{e^2}{V_0\epsilon_0}.
\end{equation}


Diagrammatically: 
\begin{equation*}
\begin{gathered}
\begin{tikzpicture}
	\begin{feynman}[large]
	\vertex [small, dot](c) at (0,0){};
	\vertex [above left=2cm of c] (i1) {};
	\vertex [below left=2cm of c] (i2){};
	\vertex [above right=2cm of c] (o1);
	\vertex [below right=2cm of c](o2);
	\diagram*{(i2)--[fermion, edge label'={$k, \sigma$}] (c) --[fermion, edge label ={$k+q, \sigma$}] (i1),
	(o2) --[fermion, edge label'={$k', \sigma'$}] (c) --[fermion, edge label={$k'-q, \sigma'$}] (o1)};
	\node[anchor = west] at (-0.4,0.2) {$\sim \tilde{V}(q)$};
%	\node[anchor = west] at ($(b)+(0.1,0)$)  {$g_{q}$};
	\end{feynman}
\end{tikzpicture}
\end{gathered}
\end{equation*}
A scattering of two electrons off one another. More generally, the interaction may be presented as follows:

\begin{align*}
	\begin{gathered}
		\begin{tikzpicture}
			\begin{feynman}
				\vertex [large, blob](c) at (0,0){};
				\vertex [above left=of c] (i1) {};
				\vertex [below left=of c] (i2){};
				\vertex [above right=of c] (o1);
				\vertex [below right=of c](o2);
				\diagram*{(i2)--[fermion] (c) --[fermion] (i1),
					(o2) --[fermion] (c) --[fermion] (o1)};
			\end{feynman}
		\end{tikzpicture}
	\end{gathered} &= 
	\begin{gathered}
		\begin{tikzpicture}
			\begin{feynman}
				\vertex [large, dot](c) at (0,0){};
				\vertex [above left=of c] (i1) {};
				\vertex [below left=of c] (i2){};
				\vertex [above right=of c] (o1);
				\vertex [below right=of c](o2);
				\diagram*{(i2)--[fermion] (c) --[fermion] (i1), (o2) --[fermion] (c) --[fermion] (o1)};
			\end{feynman}
		\end{tikzpicture}
	\end{gathered} \\ +
	\begin{gathered}
		\begin{tikzpicture}
			\begin{feynman}
				\vertex [large, dot](c) at (0,0){};
				\vertex [above left=of c] (i1) {};
				\vertex [below left=of c] (i2){};
				\vertex [large, dot, right = of c] (c2){};
				\vertex [above right=of c2] (o1);
				\vertex [below right=of c2](o2);
				\diagram*{(i2)--[fermion] (c) --[fermion] (i1),(c)--[fermion, quarter left] (c2) --[fermion, quarter left] (c),  (o2) --[fermion] (c2) --[fermion] (o1)};
			\end{feynman}
		\end{tikzpicture}
	\end{gathered} &+
	\begin{gathered}
		\begin{tikzpicture}
			\begin{feynman}
				\vertex [large, dot](c) at (0,0){};
				\vertex [above left=of c] (i1) {};
				\vertex [below left=of c] (i2){};
				\vertex [large, dot, right = of c] (c2){};
				\vertex [large, dot, right = of c2] (c3){};
				\vertex [above right=of c3] (o1);
				\vertex [below right=of c3](o2);
				\diagram*{(i2)--[fermion] (c) --[fermion] (i1),(c)--[fermion, quarter left] (c2) --[fermion, quarter left] (c),(c2)--[fermion, quarter left] (c3) --[fermion, quarter left] (c2), (o2) --[fermion] (c3) --[fermion] (o1)};
			\end{feynman}
		\end{tikzpicture}
	\end{gathered} \\
	+
	\begin{gathered}
		\begin{tikzpicture}
			\begin{feynman}
				\vertex [large, dot](c) at (0,0){};
				\vertex [above left=of c] (i1) {};
				\vertex [below left=of c] (i2){};
				\vertex [large, dot, right = of c] (c2){};
				\vertex [large, dot](ii) at ($(c)!0.5!(c2) - (0,0.5)$){};
				\vertex (iii) at (ii){};
				\vertex [above right=of c2] (o1);
				\vertex [below right=of c2](o2);
				\diagram*{(i2)--[fermion] (c) --[fermion] (i1),
					(c)--[fermion, quarter left] (c2),
					(o2) --[fermion] (c2) --[fermion] (o1),
					(c2)--[out = 225, in =0, fermion] (ii) --[fermion, out=180, in = -45] (c),
					(ii)--[fermion, out = -45, in=225, loop, min distance =1.5cm] (iii)};
			\end{feynman}
		\end{tikzpicture}
	\end{gathered} &+ \dots 
	%
	%
	%
	%
\end{align*}
We now focus on a resummation of the simplest bubble-diagrams. This is an approximation, but it will give the qualitative correct result. The diagram which we would neglect above would be the last one. Then the effective screened interaction $\tilde{V}_{\text{SC}}(q)$ would be
\begin{equation*}
	\begin{gathered}
		\feynmandiagram{a[blob]};
	\end{gathered} = 
	\begin{gathered}
		\feynmandiagram{a[dot]};
	\end{gathered} +
	\begin{gathered}
		\feynmandiagram[horizontal = a to b]{a[dot]--[fermion, quarter left] b[dot] --[fermion, quarter left] a};
	\end{gathered} +
	\begin{gathered}
		\feynmandiagram[horizontal = a to c]{b[dot]--[fermion, quarter left] a[dot] --[fermion, quarter left] b, a--[fermion, quarter left] c[dot] --[fermion, quarter left] a};
	\end{gathered} + \dots
\end{equation*}
\begin{align}
\begin{split}
	\tilde{V}_{\text{SC}}(q) &= \tilde{V}(q) + \tilde{V}(q)\Pi \tilde{V}(q) + \tilde{V}(q)\Pi \tilde{V}(q)\Pi \tilde{V}(q) +\dots \\
	&= \tilde{V}(q)\left(1 + \tilde{V}(q)\Pi + \tilde{V}(q)\Pi\tilde{V}(q)\Pi + \dots \right) \\
	&= \frac{\tilde{V}(q)}{1-\tilde{V}(q)\Pi}
\end{split}
\end{align}






\subsection{Phonon mediated electron-electron interaction}
% Start of lecture notes week 9

Due to the electron-phonon coupling depicted in \cref{fig:el-ph-vertex}, we will get an effective phonon-mediated interaction between electrons, depicted in \cref{fig:el-ph-interaction}

\begin{figure}
	\centering
	\input{tex/img/el_phonon_vertex}
	\caption{Diagram of electron-phonon vertex}
	\label{fig:el-ph-vertex}
\end{figure}


\begin{figure}
	\centering
	\input{tex/img/el_phonon_interaction}
	\caption{Diagram of the effective electron-phonon interaction}
	\label{fig:el-ph-interaction}
\end{figure}


This is an exchange of a virtual phonon. The above diagram is the effective interaction to second order in $g_q$ if we regard the wavy line \feynmandiagram{a--[boson] b,}; as a bare phonon Green's function. We could also imagine that we replaced this by
\input{tex/img/phonon_blob1}
%\feynmandiagram{a--[boson] b --[fermion, half left] c -- [fermion, half left] b, c--[boson] d,
which would include an effective interaction computed correctly up to order $\mathcal{O}(g^4)$. In fact, we might replace $D_0 + D_0\Pi D_0 +\dots$ by $D!$ Thus computing the effective interaction up to infinite order in $g$. Another, often used approach would be to replace \feynmandiagram{a--[boson] b,}; by  \todo{SETT INN DIAGRAM}

Here, we have resummed a \underline{subset} of diagrams to infinite order in $g$ in order to get an effective interaction between electrons. Under the assumption that $g$ is weak, we will keep terms only to $\mathcal{O}(g^2)$.
\begin{equation}
V_{\text{eff}}(q, \omega) = |g_q^2|\frac{2\omega_q}{\omega^2-\omega_q^2}
\end{equation}
Thus, the interaction part of the Hamiltonian becomes 
\begin{align}
\Ha &= \sum_{\stackrel{k,k',q}{\sigma,\sigma'}}V_{\text{tot}}(q, \omega)c_{k+q, \sigma}^\dagger c_{k'-q, \sigma'}^\dagger c_{k', \sigma'}c_{k,\sigma} \\
V_{\text{tot}}(q, \omega) &= \frac{e^2}{4\pi\ep q^2} + V_{\text{eff}}(q, \omega),
\end{align}

where the first term is the Coulomb-interaction. 
Furthermore, $\omega$ is the energy transfer between scattering electrons when they exchange a phonon
\begin{equation}
\omega = \ep_{k+q} - \ep_k
\end{equation}

\todo{Sett inn figur}

Note the singularities in $V_{\text{tot}}$ when $|\omega|\rightarrow\omega_q$. In particular, note the \underline{negative} singularity when $|\omega|\rightarrow\omega_q^-$. This singularity persists when Coulomb-repulsion is included. For most frequencies, the Coulomb-interaction completely dominates. However, in a narrow $\omega$-region close to $\omega_q$, the extremely weak electron-phonon coupling will always beat the Coulomb-interaction! This frequency is slightly smaller than $\omega_q$. For small $\omega$, $V_{\text{tot}}$ is repulsive. For large $\omega$, $V_{\text{tot}}$ is repulsive. For $|\omega| \lesssim \omega_q$, $V_{\text{tot}}$ is \underline{attractive}.

Let us try to give a physical picture for this: When an electron moves past an ion, they interact. The electron pulls slightly on the heavy, positively charged ion. Electrons are light, and move much faster than the heavy ions. The electron this moves quickly out of the scattering zone, while the ion relaxes slowly back to its equilibrium position. The ion in its out-of-equilibrium position represents excessive positive charge in that position, which can pull another electron towards it. This is effectively a charge-dipole interaction. If the second electron ``waits'' a little for the first electron to get away (thus reducing Coulomb-repulsion) but does not wait for too long (such that the ion has relaxed back to its equilibrium position), then the second electron can be attracted the scattering region. Effectively, the second electron is attracted to the scattering region \underline{because} the first electron was there. This is an effective electron-electron \underline{attraction}. It only works if the electron waits a little, but not for too long. 
A minimum time corresponds to a maximum frequency, while a maximum time corresponds to a minimum frequency. This implies that $V_{\text{tot}}$ is attractive if $\omega_\text{min} <\omega<\omega_\text{masx}$, as depicted in \todo{Sett in figurer og referanse til figuren (s.7)}
We may view the effective electron-electron attraction as a result of an electron locally deforming an elastic medium. Think of a rubber membrane that you put a little metal sphere on. The membrane is stretched, dipping down where you put the first sphere. If you put another little sphere on the membrane, it will fall into the dip, i.e. it will be attracted to the first particle.
This is also how gravity works: A mass deforms space-time (an elastic medium) and thus attracts another mass. 

\underline{Disclaimer}: The above two analogues are \underline{classical}. There will be an important \underline{quantum effect} coming into play here, which we will come back to. here, it will suffice to not that, classically, one can keep adding particles to the dip, such that all particles will be gathered in the same one, forming a large heavy object. This is not how it works quantum mechanically with fermions. Note also that in $V_{\text{tot}}$, and the two different simplified models for $\bar V$, they are only attractive up to a maximum $\omega$, i.e. only after a minimum amount of time. The second particle has to wait a minimum amount of time for the interaction to be attractive. This is called retardation. 
\begin{tcolorbox}
	The electrons avoid the Coulomb-interactions by avoiding each other, not in space, but in time. 
\end{tcolorbox}


\subsection[Magnons]{Magnon mediated electron-electron interaction}
% p. 10 week 9

We have seen how a boson (a phonon) with a linear coupling to electrons could give an effective attractive interaction aming electrons. 
What is we couple the electrons linearly to other bosons? One obvious thing to investigate, is to consider the coupling of electrons to \underline{magnons}.
For simplicity, we consider itinerant electrons coupled to spin-fluctuations in a \underline{ferromagnetic insulator.} (FMI) The question is if the spin-fluctuations of the FMI can give rise to an attractive interaction among electrons.
We therefore consider a system of itinerant electrons with Hamiltonian 
\begin{equation}
	\Ha_{\text{el}} = \sum_{k, \sigma}(\ep_{k}-\mu)c_{k\sigma}^\dagger c_{k\sigma}.
\end{equation}
In this system, we envisage a regular lattice of localized spins with ferromagnetic coupling, with Hamiltonian
\begin{equation}
	\Ha_{\text{spin}} = -J\sum_{\langle i,j\rangle}\vb S_i\cdot\vb S_j.
	\end{equation}
The localized spins are denoted by capital letter $\vb S$.
The coupling between the localized spins (FMI) and the itinerant electron spins $\vb s_i$ (lower case) is given by 
\begin{equation}
	\Ha_{\text{el-spin}}  = -J_{sd}\sum_i\vb S_i\cdot \vb s_i.
\end{equation}
As a minimal model, we have assumed that the electrons are hopping around on the same regular lattice that the localized spins are located. 
Using the Holstein-Primakoff transformation, ignoring the classical ground-state energy, and expressing operators in momentum space, we have 
\begin{align}
	\Ha_{\text{spin}} &= \sum_q \omega_qa_q^\dagger a_q \\
	\omega_q &= 2JS(z- \gamma(\vb q))) \\
	\gamma(\vb q) &= \sum_\delta\e^{i\vb q\cdot \vb \delta},
\end{align}
where $\vb \delta$ connects site $i$ to all its nearest neighbours.
One important fact to make note of at once, is that $\omega_q \sim q^2$ for small $q$. For the phonon-case, with acoustical phonons, $\omega_q \sim q$. Thus $\omega_q$ for small $q$ is much smaller for ferromagnetic magnons than acoustical phonons. We will return to this point. 
Consider next the electron-spin coupling: 
\begin{align}
	\Ha_\text{el-spin} &= -J_{sd}\sum_i\vb S_i\cdot \vb s_i \\
	&=-J_{sd}\sum_i(S_{iz}s_{iz} + S_{ix}s_{ix} + S_{iy}s_{iy}) \\
	&=-J_{sd}\sum_i\left(S_{iz}s_{iz}+ \frac12\left(S_{i+}s_{i-} + S_{i-}s_{i+}\right)\right),
\end{align}
where $S_{i\pm} = S_{ix}\pm iS_{iy}$,\quad \(S_{iz} = S-a_i^\dagger a_i, \quad S_{i+} = \sqrt{2S}a_i, \quad S_{i-} = \sqrt{2S}a_i^\dagger\). $\vb s_i =\frac12c_{i\alpha}^\dagger \vec\sigma_{\alpha\beta}c_{i\beta}$ with implicit summation over repeated indices $\alpha, \beta$. \[\implies s_{iz} = \frac12(\cd_{i\uparrow}c_{i\uparrow} - \cd_{i\downarrow}c_{i\downarrow}) = \frac12\sum_\sigma\sigma\cd_{i\sigma}c_{i\sigma}\]
\begin{align*}
\sigma^x = \mqty(\pmat1) &&\sigma^y = \mqty(\pmat2) &&\sigma^z= \mqty(\pmat3) \\[2ex]
\sigma^\pm = \sigma^z \pm i\sigma^y &&\sigma^+ = \begin{pmatrix}
0 & 2 \\ 0&0
\end{pmatrix}
&&\sigma^- = \begin{pmatrix}
0&0\\2&0
\end{pmatrix}
\end{align*}
Thus, we have
\begin{align}
\begin{split}
	\Ha_\text{el-spin} &= -J_{sd}S\sum_{i,\sigma}\sigma\cd_{i\sigma}c_{i\sigma}+J_{sd}\sum_{i,\sigma}\sigma a_i^\dagger a_i \cd_{i\sigma}c_{i\sigma} \\ 
	&-\frac{J_{sd}\sqrt{2S}}{2}\sum_i\left(a_i\cd_{i\downarrow}c_{i\uparrow} + \ad_i \cd_{i\uparrow}c_{i\downarrow}\right)
\end{split}
\end{align}

For the remainder of the calculation, we focus on the linear coupling of magnons to electrons, and ignore the second term. Thus, we focus on el-el interaction mediated by the vertices in \cref{fig:el-magnon-vertices}.

\begin{figure}
	\centering
	\begin{subfigure}{0.49\linewidth}
		\centering
		\input{tex/img/el_magnon_vertex1}
		\subcaption{Spin-1 magnon is dumped into electron, flipping $\downarrow\rightarrow\uparrow$}
	\end{subfigure}
	\begin{subfigure}{0.49\linewidth}
		\centering
		\input{tex/img/el_magnon_vertex2}
		\subcaption{Spin-1 magnon is excited, taking with it a spin-1, flipping $\uparrow\rightarrow\downarrow$}
	\end{subfigure}
	\caption{The two interaction vertices of interest}
	\label{fig:el-magnon-vertices}
\end{figure}

% \todo{Sett inn figurer}
 
 
 
% 
% \section*{Magnon-mediated electron-electron interactions}
% 
% We have seen how a boson (a phonon) with a linear coupling to electrons \\
% 
% \begin{center}
% 	\begin{tikzpicture}
% 	\begin{feynman}[large]
% 	\vertex (a); 
% 	\vertex[right= 3 cm of a] (b);
% 	\vertex[above right= 3cm of b] (c); 
% 	\vertex[below right= 3 cm of b] (d); 
% 	\diagram*{(a) --[photon, edge label = \(q \comma \omega \)] (b), (b) --[fermion, edge label = \( k + q \comma \sigma\)] (c), 
% 		(b) --[anti fermion, edge label = \(k \comma \sigma \)] (d),
% 	}; 
% 	\node[anchor = west] at (3.2,0) {$g_q$};
% 	\end{feynman}
% 	\end{tikzpicture}
% \end{center}
% 
% could give an effective attractive interaction among electrons. What if we couple electrons linearly to other bosons? One obvious thing to investigate, is to consider the coupling of electrons to magnons. For simplicity, we consider itinerant electrons coupled to spin-fluctuations in a ferromagnetic insulator. The question is if the spin-fluctuations of the FMI can give rise to an attractive interaction among electrons. We therefore consider a system of itinerant electrons with hamiltonian 
% 
% \begin{equation}
% \Ha_{el} = \sum_{k,\sigma} \left(\ep_k - \mu \right) \cd_{k\sigma} c_{k\sigma}.
% \end{equation}
% 
% In this system, we envisage a regular lattice of localized spins with ferromagnetic coupling, with hamiltonian 
% 
% \begin{equation}
% \Ha_{spin} = -J\sum_{\expval{i,j}} S_i \vdot S_j.
% \end{equation}
% 
% The localized spins are denoted by capital letters. The coupling between the localized spins (FMI) and the itinerant electron spins $s_i$ (denoted by lower case letters), is given by 
% 
% \begin{equation}
% \Ha_{el-spin} = - J_{sd} \sum_i S_i \vdot s_i.
% \end{equation}
% 
% As a minimal model, we have assumed that electrons are hopping around on the same regular lattice that the localized spins are located. Using the Holstein-Primakoff transformation, ignoring classical ground state energy, and expressing operators in momentumspace, we have 
% 
% \begin{align*}
% \Ha_{spin} = \sum_q \omega_q \ad_q a_q \\
% \omega_q = 2JS(z - \gamma(q)) \\
% \gamma(q) = \sum_\delta \e^{iq\vdot \delta}
% \end{align*}
% 
% $\delta$: vector connecting index $i$ to all its nearest-neighbours. \\
% 
% One important fact to make note of at once, is the fact that $\omega_q \sim q^2$ for small $q$. For the phonon case, with acoustical phonons, $\omega_q \sim q$. Thus $\omega_q$ for small $q$ is much smaller for ferromagnetic magnons than acoustical phonons. We will return to this point. \\
% 
% Consider next the electron-spin coupling: 
% 
% \begin{align*}
% \Ha_{el-spin} = -J_{sd}\sum_i S_i \vdot s_i = -J_{sd}\sum_i \left( S_{iz}s_{iz} + S_{ix}s_{ix} + S_{iy}s_{iy} \right) \\
% = -J_{sd}\sum_i \left( S_{iz}s_{iz} + \frac{1}{2}\left( S_{i+}s_{i-} + S_{i-}s_{i+} \right) \right)
% \end{align*}
% 
% where $S_{i\pm} = S_{ix} \pm iS_{iy}$. 
% 
% \begin{align*}
% S_{iz} = S - \ad_i a \quad S_{i+} \approx \sqrt{2S}a_i \quad S_{i-} \approx \sqrt{2S}\ad_i \\
% s_i = \frac{1}{2}\cd_{i\alpha}\sigma_{\alpha \beta}c_{i\beta}
% \end{align*}
% 
% where $\sigma = (\sigma^x, \sigma^y, \sigma^z)$ is a 3-tuple of Pauli-matricies and $\alpha, \beta$ are matrix indicies with Einstein convention. 
% 
% \begin{align*}
% s_{iz} = \frac{1}{2}\left(\cd_{i \uparrow} c_{i \uparrow} - \cd_{i \downarrow} c_{i \downarrow} \right) = \frac{1}{2}\sum_{i,\sigma} \sigma \cd_{i\sigma} c_{i\sigma} \\
% s_{i\pm} = \frac{1}{2}\cd_{i\alpha}\left(\sigma^x _{\alpha, \beta} \pm i\sigma^y_{\alpha, \beta}\right) c_{i\beta} \\
% s_{i+} = \frac{1}{2} \begin{pmatrix} \cd_{i \uparrow} & \cd_{i \downarrow} \end{pmatrix} \begin{pmatrix} 0 &  2 \\ 0 & 0 \end{pmatrix} \begin{pmatrix} c_{i \uparrow} \\ c_{i \downarrow} \end{pmatrix} = \cd_{i \uparrow} c_{i \downarrow} \\
% s_{i-} = \frac{1}{2} \begin{pmatrix} \cd_{i \uparrow} & \cd_{i \downarrow} \end{pmatrix} \begin{pmatrix} 0 &  0 \\ 2 & 0 \end{pmatrix} \begin{pmatrix} c_{i \uparrow} \\ c_{i \downarrow} \end{pmatrix} = \cd_{i \downarrow} c_{i \uparrow}.
% \end{align*}
% 
% Inserting these, we have 
% 
% \begin{align*}
% \Ha_{el-spin} = - J_{sd}S \sum_{i,\sigma} \sigma \cd_{i\sigma} c_{i\sigma} + J_{sd} \sum_{i,\sigma} \sigma \ad_i a_i \cd_{i\sigma} c_{i\sigma} - \frac{J_{sd}}{2}\sum_i \sqrt{2S}\left(a_i \cd_{i \downarrow} c_{i \uparrow} + \ad_i \cd_{i \uparrow} c_{i \downarrow} \right). 
% \end{align*}
% 
% For the remainder of the calculation, we focus on the linear coupling of magnons to electrons, and ignore the second term. Thus we focus on the el-el interactions mediated by the vertices 
% 
% \begin{center}
% 	\begin{tikzpicture}
% 	\begin{feynman}[large]
% 	\vertex (a); 
% 	\vertex[left= 3 cm of a] (b);
% 	\vertex[above left= 3cm of b] (c); 
% 	\vertex[below left= 3 cm of b] (d); 
% 	\diagram*{(a) --[charged boson, edge label = \(q\)] (b), (b) --[fermion, edge label' = \( k + q \comma \downarrow\)] (c), 
% 		(b) --[anti fermion, edge label = \(k \comma \uparrow\)] (d),
% 	}; 
% 	\node[anchor = west] at (-5,0) {$-\frac{J_{sd}}{2}\sqrt{2S}$};
% 	\node[anchor = west] at (1.5,0) {Magnon is annihilated, flipping spins $\uparrow \rightarrow \downarrow$};
% 	\end{feynman}
% 	\end{tikzpicture}
% \end{center}
% 
% \begin{center}
% 	\begin{tikzpicture}
% 	\begin{feynman}[large]
% 	\vertex (a); 
% 	\vertex[left= 3 cm of a] (b);
% 	\vertex[above left= 3cm of b] (c); 
% 	\vertex[below left= 3 cm of b] (d); 
% 	\diagram*{(a) --[anti charged boson, edge label = \(q\)] (b), (b) --[fermion, edge label' = \( k - q \comma \uparrow\)] (c), 
% 		(b) --[anti fermion, edge label = \(k \comma \downarrow\)] (d),
% 	}; 
% 	\node[anchor = west] at (-5,0) {$-\frac{J_{sd}}{2}\sqrt{2S}$};
% 	\node[anchor = west] at (1.5,0) {Magnon is excited, flipping spins $\downarrow \rightarrow \uparrow$};
% 	\end{feynman}
% 	\end{tikzpicture}
% \end{center}
 
 The Hamiltonian is then
 
 \begin{align*}
 H = \sum_{k,\sigma} (\ep_k - \mu - JS\sigma)\cd_{k\sigma} c_{k\sigma} + \sum_q \omega_q \ad_q a_q - \frac{J_{sd}}{2}\sqrt{2S}\sum_{k,q} \left( a_q \cd_{k+q,\downarrow} c_{k\uparrow} + \ad_q \cd_{k-q,\uparrow} c_{k,\downarrow}\right).
 \end{align*}
 
 The effective interaction mediated by this coupling, will be to second order in $J_{sd}$:
 
 \begin{center}
 	\begin{tikzpicture}
 	\begin{feynman}[large]
 	\vertex (a); 
 	\vertex[left= 3 cm of a] (b);
 	\vertex[above left= 3cm of b] (c); 
 	\vertex[below left= 3 cm of b] (d); 
 	\vertex[above right= 3cm of a] (f1); 
 	\vertex[below right= 3cm of a] (f2); 
 	\diagram*{(a) --[charged boson, edge label = \(q\)] (b), (b) --[fermion, edge label' = \( k + q \comma \downarrow\)] (c), 
 		(b) --[anti fermion, edge label = \(k \comma \uparrow\)] (d), (a) --[fermion, edge label = \(k' - q \comma \uparrow\)] (f1), (a) --[anti fermion, edge label' = \(k' \comma \downarrow\)] (f2),
 	}; 
 	\node[anchor = west] at (-5,0) {$-\frac{J_{sd}}{2}\sqrt{2S}$};
 	\node[anchor = west] at (0.2,0) {$-\frac{J_{sd}}{2}\sqrt{2S}$};
 	\end{feynman}
 	\end{tikzpicture}
 \end{center}
 
 \begin{equation}
 V_{eff}(q,\omega) = \left(-\frac{J_{sd}}{2}\sqrt{2S}\right)^2 D_0(q,\omega)
 \end{equation}
 
 $D_0(q,\omega)$: Free magnon Green's function. Note that $\omega = \ep_{k+q \downarrow} - \ep_{k\uparrow} = \ep_{k' \downarrow} - \ep_{k' - q\uparrow}$, where $\ep_{k\sigma} = \ep_k - \mu - JS\sigma$. Thus 
 
 \begin{align}
 \omega = \ep_{k+q} - \ep_k +2JS = \ep_{k'} - \ep_{k'-q} + 2JS.
 \end{align}
 
 $2JS$ acts like a magnetic field, $q \to 0 \rightarrow \abs{\omega} \to 2JS$. This fact also suppresses attractive interactions, since it increases the denominator of $V_{eff}$. 
 
 \begin{align*}
 D_0(q, \omega) = -i \bra{\varphi_0} T[a_q(t) \ad(t')]\ket{\varphi_0} \\ 
 a_q(t) = \e^{i\Ha_0 t} a_q \e^{-i \Ha_0 t} \\
 \ad_q = \e^{i\Ha_0 t'} \ad_q \e^{-i \Ha_0 t'} \\
 \Ha_0 = \sum_q \omega_q \ad_q a_q
 \end{align*}
 
 where time-evolution is expressed in the interaction picture. Formally, this is exactly the same as for the phonon Green's function, and hence 
 
 \begin{equation}
 V_{eff}(q,\omega) = \left(-\frac{J_{sd}}{2}\sqrt{2S}\right)^2 \frac{2\omega_q ^2 }{\omega^2 - \omega_q ^2},
 \end{equation}
 
 which is an attractive interaction if $\abs{\omega} < \abs{\omega_q}$. \\
 
 Recall what we found with phonons: phonon-mediated el-el interactions was found to be 
 
 \begin{equation}
 V_{eff}(q,\omega) = \abs{g_q}^2 \frac{2\omega_q ^2 }{\omega^2 - \omega_q ^2}.
 \end{equation}
 
 There are a couple of notable differences between these two results: 
 
 \begin{itemize}
 	\item The coupling constant $-\frac{J_{sd}}{2}\sqrt{2S}$ for magnons is constant for $q \to 0$. The coupling constant $g_q$ for phonons go to 0 as $q \to 0$. For optical phonons, it vanishes $\sim q$. For acoustical phonons it vanishes $\sim \sqrt{q}$. The acoustical phonons are thus more important for creating a phonon mediated el-el interaction. 
 	\item Acoustical phonons: $\omega_q \sim \abs{q} \quad (\sim 1 THz)$ \\ Magnons: $\omega_q \sim q^2 \quad (\sim 1GHz)$ \\ Thus, for small $q$, $\omega_q ^{mag} \ll \omega_q ^{ph}$. 
 	
 \end{itemize}
 
 In total, we thus have 
 
 \begin{align*}
 V_{eff}^{\text{mag}} \sim \frac{q^4}{\omega^2 - c_1 q^4} &&\text{ Ferromagnetic magnons}\\ 
 V_{eff}^{\text{ph}} \sim \frac{q^3}{\omega^2 - c_2 q^4} && \text{Acoustic phonons}
 \end{align*}
 
 Thus,  while ferromagnetic magnons will be able to create an attractive interaction between electrons, this attraction will be much weaker (due to smallness of the coupling constant and argument above) than the attraction created by phonons. \\
 
 A good strategy for finding strong electron-electron attractions mediated by some boson, would be to look for bosons giving rise to a 
 
 \begin{equation}
 V_{eff}(q,\omega) = \abs{\lambda_q}^2 \frac{2\omega_q ^2 }{\omega^2 - \omega_q ^2},
 \end{equation} 
 
 with  $\omega_q \sim \abs{q}$ and $\lim_{q \to 0} \lambda_q = \lambda_0 \neq 0$. Candidate: Anti-ferromagnetic magnons! \\
 
 Next, we will consider a very simple problem to illustrate the dramatic effect that an attractive interaction among electron has. The problem is so simple that it can be solved exactly. An important point to note, is that the solution to the problem will demonstrate that the answer could not have been found to any finite order in perturbation theory, no matter how weak the interaction is. Attractive interactions in the electron-system is a singular perturbation! 
 
 